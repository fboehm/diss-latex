\section{Overview of statistics}

One characterization of statistics is that it is the science of learning
about natural phenomena. The process of learning about natural phenomena proceeds
in a cyclical fashion. One first posits a ``model'' for a natural phenomenon.
She designs an experiment to measure a quantity related to the phenomenon of interest.
She then collects data, summarizes it as a statistic, and performs a test of
competing hypotheses before updating her hypothesis about the natural phenomenon.
The newly updated hypothesis replaces the original hypothesis in the second
iteration of the cycle. 

For example, we might wish to study the genetics of body weight in mice. 
We first posit the existence of body weight quantitative trait loci (QTL), which
are regions of the genome that affect mouse body weight. We quantitatively state the competing possibilities as the null hypothesis that there is no QTL against the alternative hypothesis of presence of a QTL. We then design an experiment, perhaps using genetically diverse mice, like those from the Diversity Outbred mouse population. We measure body weight in, say, 100 Diversity Outbred mice and obtain genome-wide genetic data from each. We then summarize our observations by performing multiple instances of a statistical technique called ``linear regression''. We perform statistical hypothesis tests to quantify the evidence against the null hypothesis.  

\section{Maximum likelihood methods}




\section{Restricted maximum likelihood methods}

