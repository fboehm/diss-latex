\newenvironment{abstract}%
    {\cleardoublepage\null\vfill\begin{center}%
    \bfseries\abstractname\end{center}}%
    {\vfill\null}
        \begin{abstract}
I develop a test of pleiotropy vs. separate QTL for multiparental populations.
It extends the two-parent cross methods of \citet{jiang1995multiple} by
surmounting three challenges in the multiparental population setting. 
First, my test accommodates more than two founder alleles.
Second, my test incorporates multivariate polygenic random effects to allow
for complicated pedigrees. 
Third, it uses a bootstrap test to get p-values. 

After discussing methods in Chapter 2, I demonstrate its use in analyses of
data from Diversity Outbred mice in Chapter 3. 
I organize Chapter 3 as three vignettes. 
Each allows us to study properties and applications of the test.
First, I perform a comparative study of my pleiotropy test with 
linear regression-based mediation analysis in the context of dissecting
an expression trait hotspot.
Second, I examine the test's power through a study of local expression QTL. 
Lastly, I apply the test to inform gut microbiome studies.

In Chapter 4, I present a tutorial on the software package \texttt{qtl2pleio}. 
\texttt{qtl2pleio} is a R package that implements my test and related
analyses and visualizations. 

I offer concluding thoughts in Chapter 5. 
\end{abstract}