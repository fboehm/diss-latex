\documentclass[oneside]{book}
\usepackage[utf8]{inputenc}
\usepackage[colorinlistoftodos]{todonotes}
\usepackage{setspace}
\usepackage[left = 1in, right = 1in, top = 1.5in, bottom = 1in]{geometry}
\usepackage{multirow}
\usepackage[framemethod=TikZ]{mdframed}
\usepackage{newfloat}
\usepackage{multicol}
\usepackage[titletoc]{appendix}
\usepackage{amsmath}
\usepackage[
    backend=biber,
    style=authoryear,
    natbib=true,
    url=true, 
    doi=true,
    eprint=true
]{biblatex}
\addbibresource{research.bib}
\usepackage{graphicx}
\usepackage{subcaption}
\usepackage{array}
\usepackage{amssymb}
\usepackage{amsthm}
\usepackage{amsfonts}
\usepackage{longtable}
\usepackage[outdir=./]{epstopdf}
\usepackage{fancyhdr}
\setlength{\headheight}{15.2pt}
\pagestyle{fancyplain}
\fancyhf{}
\lhead{}
\chead{}
%\rhead{}
\rhead{ \fancyplain{}{\thepage} }
\renewcommand{\headrulewidth}{0pt} % remove top line
\renewcommand{\footrulewidth}{0pt} % remove bottom line
\DeclareCaptionType{equ}[][] 
%https://stackoverflow.com/questions/149479/adding-a-caption-to-an-equation-in-latex
%\captionsetup[equ]{labelformat=empty}

%%%%%% VIGNETTE HEADER STUFF
\usepackage{lmodern}
%\usepackage{amssymb,amsmath}
\usepackage{ifxetex,ifluatex}
\usepackage{fixltx2e} % provides \textsubscript
\ifnum 0\ifxetex 1\fi\ifluatex 1\fi=0 % if pdftex
  \usepackage[T1]{fontenc}
  \usepackage[utf8]{inputenc}
\else % if luatex or xelatex
  \ifxetex
    \usepackage{mathspec}
  \else
    \usepackage{fontspec}
  \fi
  \defaultfontfeatures{Ligatures=TeX,Scale=MatchLowercase}
\fi
% use upquote if available, for straight quotes in verbatim environments
\IfFileExists{upquote.sty}{\usepackage{upquote}}{}
% use microtype if available
\IfFileExists{microtype.sty}{%
\usepackage{microtype}
\UseMicrotypeSet[protrusion]{basicmath} % disable protrusion for tt fonts
}{}
%\usepackage[margin=1in]{geometry}
%\usepackage{hyperref}
%\hypersetup{unicode=true,
%            pdftitle={Recla Analysis},
%            pdfauthor={Frederick Boehm},
%            pdfborder={0 0 0},
%            breaklinks=true}
%\urlstyle{same}  % don't use monospace font for urls
\usepackage{color}
\usepackage{fancyvrb}
\newcommand{\VerbBar}{|}
\newcommand{\VERB}{\Verb[commandchars=\\\{\}]}
\DefineVerbatimEnvironment{Highlighting}{Verbatim}{commandchars=\\\{\}}
% Add ',fontsize=\small' for more characters per line
\usepackage{framed}
\definecolor{shadecolor}{RGB}{248,248,248}
\newenvironment{Shaded}{\begin{snugshade}}{\end{snugshade}}
\newcommand{\AlertTok}[1]{\textcolor[rgb]{0.94,0.16,0.16}{#1}}
\newcommand{\AnnotationTok}[1]{\textcolor[rgb]{0.56,0.35,0.01}{\textbf{\textit{#1}}}}
\newcommand{\AttributeTok}[1]{\textcolor[rgb]{0.77,0.63,0.00}{#1}}
\newcommand{\BaseNTok}[1]{\textcolor[rgb]{0.00,0.00,0.81}{#1}}
\newcommand{\BuiltInTok}[1]{#1}
\newcommand{\CharTok}[1]{\textcolor[rgb]{0.31,0.60,0.02}{#1}}
\newcommand{\CommentTok}[1]{\textcolor[rgb]{0.56,0.35,0.01}{\textit{#1}}}
\newcommand{\CommentVarTok}[1]{\textcolor[rgb]{0.56,0.35,0.01}{\textbf{\textit{#1}}}}
\newcommand{\ConstantTok}[1]{\textcolor[rgb]{0.00,0.00,0.00}{#1}}
\newcommand{\ControlFlowTok}[1]{\textcolor[rgb]{0.13,0.29,0.53}{\textbf{#1}}}
\newcommand{\DataTypeTok}[1]{\textcolor[rgb]{0.13,0.29,0.53}{#1}}
\newcommand{\DecValTok}[1]{\textcolor[rgb]{0.00,0.00,0.81}{#1}}
\newcommand{\DocumentationTok}[1]{\textcolor[rgb]{0.56,0.35,0.01}{\textbf{\textit{#1}}}}
\newcommand{\ErrorTok}[1]{\textcolor[rgb]{0.64,0.00,0.00}{\textbf{#1}}}
\newcommand{\ExtensionTok}[1]{#1}
\newcommand{\FloatTok}[1]{\textcolor[rgb]{0.00,0.00,0.81}{#1}}
\newcommand{\FunctionTok}[1]{\textcolor[rgb]{0.00,0.00,0.00}{#1}}
\newcommand{\ImportTok}[1]{#1}
\newcommand{\InformationTok}[1]{\textcolor[rgb]{0.56,0.35,0.01}{\textbf{\textit{#1}}}}
\newcommand{\KeywordTok}[1]{\textcolor[rgb]{0.13,0.29,0.53}{\textbf{#1}}}
\newcommand{\NormalTok}[1]{#1}
\newcommand{\OperatorTok}[1]{\textcolor[rgb]{0.81,0.36,0.00}{\textbf{#1}}}
\newcommand{\OtherTok}[1]{\textcolor[rgb]{0.56,0.35,0.01}{#1}}
\newcommand{\PreprocessorTok}[1]{\textcolor[rgb]{0.56,0.35,0.01}{\textit{#1}}}
\newcommand{\RegionMarkerTok}[1]{#1}
\newcommand{\SpecialCharTok}[1]{\textcolor[rgb]{0.00,0.00,0.00}{#1}}
\newcommand{\SpecialStringTok}[1]{\textcolor[rgb]{0.31,0.60,0.02}{#1}}
\newcommand{\StringTok}[1]{\textcolor[rgb]{0.31,0.60,0.02}{#1}}
\newcommand{\VariableTok}[1]{\textcolor[rgb]{0.00,0.00,0.00}{#1}}
\newcommand{\VerbatimStringTok}[1]{\textcolor[rgb]{0.31,0.60,0.02}{#1}}
\newcommand{\WarningTok}[1]{\textcolor[rgb]{0.56,0.35,0.01}{\textbf{\textit{#1}}}}
\usepackage{graphicx,grffile}
\makeatletter
\def\maxwidth{\ifdim\Gin@nat@width>\linewidth\linewidth\else\Gin@nat@width\fi}
\def\maxheight{\ifdim\Gin@nat@height>\textheight\textheight\else\Gin@nat@height\fi}
\makeatother
% Scale images if necessary, so that they will not overflow the page
% margins by default, and it is still possible to overwrite the defaults
% using explicit options in \includegraphics[width, height, ...]{}
%\setkeys{Gin}{width=\maxwidth,height=\maxheight,keepaspectratio}
%\IfFileExists{parskip.sty}{%
%\usepackage{parskip}
%}{% else
%\setlength{\parindent}{0pt}
%\setlength{\parskip}{6pt plus 2pt minus 1pt}
%}
%\setlength{\emergencystretch}{3em}  % prevent overfull lines
%\providecommand{\tightlist}{%
%  \setlength{\itemsep}{0pt}\setlength{\parskip}{0pt}}
%\setcounter{secnumdepth}{0}
% Redefines (sub)paragraphs to behave more like sections
%\ifx\paragraph\undefined\else
%\let\oldparagraph\paragraph
%\renewcommand{\paragraph}[1]{\oldparagraph{#1}\mbox{}}
%\fi
%\ifx\subparagraph\undefined\else
%\let\oldsubparagraph\subparagraph
%\renewcommand{\subparagraph}[1]{\oldsubparagraph{#1}\mbox{}}
%\fi

%%% Use protect on footnotes to avoid problems with footnotes in titles
%\let\rmarkdownfootnote\footnote%
%\def\footnote{\protect\rmarkdownfootnote}

%%% Change title format to be more compact
%\usepackage{titling}

% Create subtitle command for use in maketitle
%\newcommand{\subtitle}[1]{
%  \posttitle{
%    \begin{center}\large#1\end{center}
%    }
%}



%%%%%% END VIGNETTE HEADER STUFF


\usepackage{hyperref}
\hypersetup{
    colorlinks=true
}
\DeclareFloatingEnvironment[fileext=frm,placement={!ht},name=Frame]{myfloat}
\captionsetup[myfloat]{labelfont=bf}
\newenvironment{frameenv}[1]
    {\begin{myfloat}[tb]
    \begin{mdframed}[roundcorner=10pt,backgroundcolor=blue!10]
    \caption{#1}
    %\begin{multicols*}{2}
    }
    {%\end{multicols*}
    \end{mdframed}\end{myfloat}
    }




\title{Testing Pleiotropy vs. Separate QTL in Multiparental Populations}
\author{Frederick Boehm}
\date{\today}



\begin{document}
\frontmatter % use roman numerals for page numbering of first pages
\doublespacing
\maketitle
\tableofcontents
%\pagestyle{headings} % page numbers appear at bottom of page, centered!
%\markright{\hfill}
\mainmatter % use arabic numerals for body of text

\chapter{Complex traits in multiparental populations}


%%% chapter 2 start
\chapter{Methods development}

\section{Introduction}

Complex trait studies in multiparental populations present new
challenges in statistical methods and data analysis. Among these is
the development of strategies for multivariate trait analysis. The
joint analysis of two or more traits allows one to address additional
questions, such as whether two traits share a single pleiotropic
locus.




Previous research addressed the question of pleiotropy vs.\ separate
QTL in two-parent crosses.
\citet{jiang1995multiple} developed a likelihood
ratio test for pleiotropy vs.\ separate QTL for a pair of traits.
Their approach assumed that each trait was affected by a single QTL.
Under the null hypothesis, the two traits were affected by a common
QTL, and under the alternative hypothesis the two traits were affected
by distinct QTL.
\citet{knott2000multitrait} used linear regression to develop a fast
approximation to the test of \citet{jiang1995multiple}, while
\citet{tian2016dissection} used the methods from
\citet{knott2000multitrait} to dissect QTL hotspots in a F$_2$
population.




Multiparental populations, such
as the Diversity Outbred (DO)mouse population \citep{churchill2012diversity}, enable high-precision
mapping of complex traits \citep{de2014genetics}. The DO
mouse population began with progenitors of the Collaborative
Cross (CC) mice \citep{churchill2004collaborative}
Each DO mouse is a highly heterozygous genetic mosaic
of alleles from the eight CC founder lines. Random
matings among non-siblings have maintained the DO
population for more than 23 generations \citep{chesler2016diversity}.

Several limitations of previous pleiotropy vs.\ separate QTL tests
prevent their direct application in multiparental populations. First,
multiparental populations can have complex patterns of relatedness
among subjects, and failure to account for these patterns of
relatedness may lead to spurious results \citep{yang2014advantages}.
Second, previous tests allowed for only two founder lines
\citep{jiang1995multiple}. Finally, \citet{jiang1995multiple} assumed
that the null distribution of the test statistic follows a chi-square
distribution.

We developed a pleiotropy vs.\ separate QTL test for two traits in
multiparental populations. Our test builds on research that
\citet{jiang1995multiple}, \citet{knott2000multitrait},
\citet{tian2016dissection}, and \citet{zhou2014efficient} initiated.
Our innovations include the accommodation of $k$ founder alleles per
locus (compared to the traditional two founder alleles per locus) and
the incorporation of multivariate polygenic random effects to account
for relatedness. Furthermore, we implemented a parametric bootstrap to
calibrate test statistic values \citep{efron1979,tian2016dissection}.

Below, we describe our likelihood ratio test for pleiotropy vs.\
separate QTL. In simulation studies, we find that it is slightly
conservative, and that it has power to detect two separate loci when
the univariate LOD peaks are strong. We further illustrate our
approach with an application to data on a pair of behavior traits in
a population of 261 DO mice \citep{logan2013high,recla2014precise}.
We find modest evidence for distinct QTL in a 2.5-cM region on mouse
Chromosome 8.


\section{Methods}
\label{sec:materials:methods}

Our strategy involves first identifying two traits that map to a common
genomic region. We then perform a two-dimensional, two-QTL scan over
the genomic region, with each trait affected by one QTL of varying
position. We identify the QTL position that maximizes the likelihood
under pleiotropy (that is, along the diagonal where the two QTL are at
a common location), and the ordered pair of positions that maximizes
the likelihood under the model where the two QTL are allowed to be
distinct. The logarithm of the ratio of the two likelihoods is our
test statistic. We calibrate this test statistic with a parametric
bootstrap.

\subsection{Data structures}

The data consist of three objects. The first is an $n$ by $k$ by $m$
array of allele probabilities for $n$ subjects with $k$ alleles and
$m$ marker positions on a single chromosome [derived from the observed
SNP genotype data by a hidden Markov model; see
\citet{broman2019rqtl2}]. The second object is an $n$ by 2 matrix of
phenotype values. Each column is a phenotype and each row is a
subject. The third object is an $n$ by $c$ matrix of covariates, where
each row is a subject and each column is a covariate.

One additional object is the genotype-derived kinship matrix, which is
used in the linear mixed model to account for population structure. We
are focusing on a defined genomic interval, and we prefer to use a
kinship matrix derived by the ``leave one chromosome out'' (LOCO)
method \citep{yang2014advantages}, in which the kinship matrix is
derived from the genotypes for all chromosomes except the chromosome
under test.




\subsection{Statistical Models}

Focusing on a pair of traits and a particular genomic region of
interest, the next step is a two-dimensional, two-QTL
scan \citep{jiang1995multiple}. We consider two QTL with each
affecting a different trait, and consider all possible pairs of
locations for the two QTL. For each pair of positions, we fit
the multivariate linear mixed effects model defined in Equation
\ref{eqn:model1}. Note that we have
assumed an additive genetic model throughout our analyses, but
extensions to design matrices that include dominance are
straightforward.


\begin{equation}
vec(Y) = X vec(B) + vec(G) + vec(E)
\label{eqn:model1}
\end{equation}
where $Y$ is the $n$ by $2$ matrix of phenotypes values;
$X$ is a $2n$ by $2(k + c)$
matrix that contains the $k$ allele probabilities for the two QTL
positions and the $c$
covariates in diagonal blocks; $B$ is a $(k + c)$ by $2$ matrix of
allele effects and covariate effects; $G$ is a $n$ by $2$ matrix of
random effects; and $E$ is a $n$ by $2$ matrix of random errors. $n$
is the number of mice. The `vec' operator stacks columns from a matrix
into a single vector. For example, a 2 by 2 matrix inputted to `vec'
results in a vector with length 4. Its first two entries are the
matrix's first column, while the third and fourth entries are the
matrix's second column.


We also impose distributional assumptions on $G$ and $E$:

\begin{equation}
G \sim MN_{n x 2}(0, K, V_g)
\label{eqn:model2}
\end{equation}

and

\begin{equation}
E \sim MN_{nx2}(0, I, V_e)
\label{eqn:model3}
\end{equation}
where $MN_{n x 2}(0, V_r, V_c)$ denotes the matrix-variate ($n$ by 2)
normal distribution with mean being the $n$ by $2$ matrix with all
zero entries and row covariance $V_r$ and column covariance $V_c$. We
assume that $G$ and $E$ are independent.


\subsection{Parameter inference and log likelihood calculation}

Inference for parameters in multivariate linear mixed effects models
is notoriously difficult and can be computationally intense
\citep{meyer1989restricted,meyer1991estimating}. Thus, we estimate
$V_g$ and $V_e$ under the null hypothesis of no QTL, and then take
them as fixed and known in our two-dimensional, two-QTL genome scan.
We use restricted maximum likelihood methods to fit the
model:

\begin{equation}
vec(Y) = X_0vec(B) + vec(G) + vec(E)
\label{model}
\end{equation}
where $X_0$ is a $2n$ by $2(c + 1)$ matrix whose first column of each
diagonal block in $X_0$ has all entries equal to one (for an intercept); the remaining
columns are the covariates.

We draw on our R implementation \citep{gemma2} of the
GEMMA algorithm for fitting a multivariate linear mixed effects model
with expectation-maximization \citep{zhou2014efficient}. We use
restricted maximum likelihood fits for the variance components $V_g$
and $V_e$ in subsequent calculations of the generalized least squares
solution $\hat B$.

\begin{equation}
    \hat B = (X^T\hat\Sigma^{-1}X)^{-1}X^T\hat\Sigma^{-1}vec(Y)
\end{equation}

\noindent where

\begin{equation}
    \hat\Sigma = \hat V_g \otimes K + \hat V_e \otimes I_n
    \label{cov}
\end{equation}

\noindent where $\otimes$ denotes the Kronecker product, $K$ is the
kinship matrix, and $I_n$ is a n by n
identity matrix. We then calculate the log likelihood for a normal
distribution with mean $X vec(\hat B)$ and covariance $\hat \Sigma$
that depends on our estimates of $V_g$ and $V_e$ (Equation \ref{cov}).

\subsection{Pleiotropy vs.\ separate QTL hypothesis testing framework}

Our test applies to two traits considered simultaneously. Below,
$\lambda_1$ and $\lambda_2$ denote putative locus positions for traits
one and two. We quantitatively state the competing hypotheses for our
test as:

\begin{eqnarray}
H_0: \lambda_1 = \lambda_2 \nonumber\\
H_A: \lambda_1 \neq \lambda_2
\label{eqn:hypotheses}
\end{eqnarray}

\noindent Our likelihood ratio test statistic is:

\begin{equation}
\text{LOD} = \log_{10} \left[ \frac{\max_{\lambda_1, \lambda_2} L(B, \Sigma, \lambda_1, \lambda_2)}{
    \max_\lambda L(B, \Sigma, \lambda, \lambda)} \right]
\label{eqn:test-statistic}
\end{equation}
where $L$ is the likelihood for fixed QTL positions,
maximized over all other parameters.

\subsection{Visualizing profile LOD traces}

The output of the above analysis is a two-dimensional log$_{10}$ likelihood
surface. To visualize these results, we followed an innovation of \citet{zeng2000genetic} and
\citet{tian2016dissection}, and plot three traces: the results along the
diagonal (corresponding to the null hypothesis of pleiotropy), and
then the profiles derived by fixing one QTL's position
and maximizing over the other QTL's position.

We define the LOD score for our test:

\begin{equation}
\text{LOD}(\lambda_1, \lambda_2) = ll_{10}(\lambda_1, \lambda_2) - \max ll_{10}(\lambda, \lambda)
\label{eq:lodpvl}
\end{equation}
where $ll_{10}$ denotes log$_{10}$ likelihood.

We follow \citet{zeng2000genetic} and \citet{tian2016dissection} in
defining profile LOD by the equation

\begin{equation}
\text{profile LOD}_1(\lambda_1) = \max_{\lambda_2}\text{LOD}(\lambda_1, \lambda_2)
\label{eq:profilelod}
\end{equation}
We define profile LOD$_2(\lambda_2)$ analogously.
The maximum value for the profile LOD$_1$
profile LOD$_2$ traces are the same and are non-negative, and give the
overall LOD test statistic.

We construct the pleiotropy trace by calculating the log-likelihoods
for the pleiotropic models at every position.

\begin{equation}
LOD_{p}(\lambda) = ll_{10}(\lambda, \lambda) - \max ll_{10}(\lambda, \lambda)
\label{eq:lodp}
\end{equation}
By definition, the maximum value for this pleiotropy trace
is zero.






\subsection{Bootstrap for test statistic calibration}

We use a parametric bootstrap to calibrate our test statistic
\citep{efron1979}. While \citet{jiang1995multiple} used quantiles of a
chi-squared distribution to determine p-values, this does not account
for the two-dimensional search over QTL positions.
We follow the approach of \citet{tian2016dissection}, and identify
the maximum likelihood estimate of the QTL position under the null
hypothesis of pleiotropy.
We then use the inferred model parameters under that model and with
the QTL at that position to simulate bootstrap data sets according to
the model in equations \ref{eqn:model1}--\ref{eqn:model3}.
For each of $b$ bootstrap data sets, we
perform a two-dimensional QTL scan (over the genomic region of
interest) and derive the test
statistic value. We treat these $b$ test statistics as the
empirical null distribution, and calculate a p-value as the
proportion of the $b$ bootstrap test statistics that equal or exceed
the observed one, with the original data,
$p = \# \{ i:\text{LOD}^*_i \geq \text{LOD}\} / b$
where $\text{LOD}_i^*$ denotes the LOD score for the $i$th bootstrap
replicate and LOD is the observed test statistic.



\subsection{Data \& Software Availability}

Our methods have been implemented in an R package, \texttt{qtl2pleio},
available at GitHub:

\href{https://github.com/fboehm/qtl2pleio}{https://github.com/fboehm/qtl2pleio}

\noindent Custom R code for our analyses and simulations are at GitHub:

\href{https://github.com/fboehm/qtl2pleio-manuscript}{https://github.com/fboehm/qt2pleio-manuscript}

\noindent The data from \citet{recla2014precise} and
\citet{logan2013high} are available at the Mouse Phenome Database:

\href{https://phenome.jax.org/projects/Chesler4}{https://phenome.jax.org/projects/Chesler4} and \href{https://phenome.jax.org/projects/Recla1}{https://phenome.jax.org/projects/Recla1}.

\noindent They are also available in R/qtl2 format at
\href{https://github.com/rqtl/qtl2data}{https://github.com/rqtl/qtl2data}.




\section{Simulation studies}

We performed two types of simulation studies, one for type I error
rate assessment and one to characterize the power to detect separate
QTL. To simulate traits, we specified $X$, $B$, $V_g$, $K$, and $V_e$
matrices (Equations \ref{eqn:model1}--\ref{eqn:model3}). For both we
used the allele probabilities from a single genomic region derived
empirically from data for a set of 479 Diversity Outbred mice from
\citet{keller2018genetic}.

\subsection{Type I error rate analysis}

To quantify type I error rate ({\em i.e.}, false positive rate), we
simulated 400 pairs of traits for each of eight sets of parameter
inputs (Table~\ref{table-typeI}). We used a $2^3$ factorial
experimental design with three factors: allele effects difference,
allele effects partitioning, and genetic correlation, \textit{i.e.},
the off-diagonal entry in the 2 by 2 matrix $V_g$.

\begin{table}
\begin{center}
  \caption{Type I error rates for all runs in our $2^3$
    experimental design. We set (marginal) genetic variances
    (\emph{i.e.}, diagonal elements of $V_g$) to 1 in all runs. $V_e$
    was set to the 2 by 2 identity matrix in all runs. We used allele
    probabilities at a single genetic marker to simulate traits for
    all eight sets of parameter inputs. In the
    column ``Allele effects partitioning'', ``ABCD:EFGH'' means that lines
    A--D carry one QTL allele while lines E--H carry the other allele.
    ``F:ABCDEGH'' means the QTL has a private allele in strain F.}
  \label{table-typeI}

  \bigskip

\small
  \begin{tabular}{ c | c | c | c | c}
    \hline
    Run & $\Delta$(Allele effects) & Allele effects partitioning & Genetic correlation & Type I error rate \\ \hline
    1 & 6 & ABCD:EFGH & 0 & 0.032\\
    2 & 6 & ABCD:EFGH & 0.6 & 0.035\\
    3 & 6 & F:ABCDEGH & 0 & 0.040\\
    4 & 6 & F:ABCDEGH & 0.6 & 0.045\\
    5 & 12 & ABCD:EFGH & 0 & 0.038\\
    6 & 12 & ABCD:EFGH & 0.6 & 0.042\\
    7 & 12 & F:ABCDEGH & 0 & 0.025\\
    8 & 12 & F:ABCDEGH & 0.6 & 0.025\\
    \hline
  \end{tabular}
\end{center}
  \end{table}

We chose two strong allele effects difference values, 6 and 12. These
ensured that the univariate phenotypes mapped with high LOD scores to
the region of interest. For the allele partitioning factor, we used
either equally frequent QTL alleles, or a private allele in the CAST
strain (F). For the residual genetic correlation (the off-diagonal
entry in $V_g$), we considered the values 0 and 0.6. The marginal
genetic variances (\textit{i.e.}, the diagonal entries in $V_g$) for
each trait were always set to one.

We performed 400 simulation replicates per set of parameter inputs,
and each used $b = 400$ bootstrap samples. For each bootstrap sample, we calculated the
test statistic (Equation \ref{eqn:test-statistic}). We then compared
the test statistic from the simulated trait against the empirical
distribution of its 400 bootstrap test statistics. When the simulated
trait's test statistic exceeded the 0.95 quantile of the empirical
distribution of bootstrap test statistics, we rejected the null
hypothesis. We observed that the test is slightly conservative over
our range of parameter selections (Table~\ref{table-typeI}), with
estimated type I error rates $<$ 0.05.


\subsection{Power analysis}

We also investigated the power to detect the presence of two
distinct QTL. We used a 2 $\times$ 2 $\times$ 5 experimental design, where our
three factors were allele effects difference, allele effects
partitioning, and inter-locus distance. The two levels of allele
effects difference were 1 and 2. The two levels of allele effects
partitioning were as in the type I error rate studies, ABCD:EFGH and
F:ABCDEGH (Table~\ref{table-letters}). The five levels of interlocus
distance were 0, 0.5, 1, 2, and 3 cM. $V_g$ and $V_e$ were both set to
the 2 by 2 identity matrix in all power study simulations.

We simulated 400 pairs of traits per set of parameter inputs. For
each simulation replicate, we calculated the likelihood ratio test
statistic. We then applied our parametric bootstrap to calibrate the
test statistics. For each simulation replicate, we used $b = 400$ bootstrap
samples. Because the bootstrap test statistics within a single set of
parameter inputs followed approximately the same distribution, we
pooled the $400 * 400 = 160,000$ bootstrap samples per set of
parameter inputs and compared each test statistic to the empirical
distribution derived from the 160,000 bootstrap samples. However, for
parameter inputs with interlocus distance equal to zero, we didn't
pool the 160,000 bootstrap samples; instead, we proceeded by
calculating power (\textit{i.e.}, type I error rate, in this case), as we did in the
type I error rate study above.

\begin{figure}
\includegraphics[width = \textwidth]{../qtl2pleio-manuscript/R/power-curves.eps}
\caption{Pleiotropy vs.\ separate QTL power curves for each of four
  sets of parameter settings. Factors that differ among the four
  curves are allele effects difference and allele partitioning. Red denotes high allele effects difference, while black is the low allele effects difference. Solid line denotes the even allele partitioning (ABCD:EFGH), while dashed line denotes the uneven allele partitioning (F:ABCDEGH).}
\label{fig:power}
\end{figure}

We present our power study results in Figure~\ref{fig:power}.
Power increases as interlocus distance increases. The top two curves
correspond to the case where the QTL effects are largest. For each value
for the QTL effect, power is greater when the QTL alleles are equally
frequent, and smaller when a QTL allele is private to one strain. One
can have high power to detect that the two traits have distinct QTL
when they are separated by $>$ 1~cM and when the QTL have large effect.


\section{Application}
\label{sec:app}

To illustrate our methods, we applied our test to data from
\citet{logan2013high} and \citet{recla2014precise}, on 261 DO mice
measured for a set of behavioral phenotypes.
\citet{recla2014precise} identified \textit{Hydin} as the gene that
underlies a QTL on Chromosome 8 at 57 cM for the ``hot plate latency''
phenotype (a measure of pain tolerance). The phenotype ``percent time in light''
in a light-dark box (a measure of anxiety) was
measured on the same set of mice \citep{logan2013high} and also shows a QTL near
this location, which led us to ask whether the same locus affects both traits.
The two traits show a correlation of $-0.15$ (Figure~\ref{fig:scatter}).

QTL analysis with the LOCO method, and using sex as an additive
covariate, showed multiple suggestive QTL for each
phenotype (Figure~\ref{fig:genomewide10-22}; Table~\ref{table-peaks}). For our investigation of
pleiotropy, we focused on the interval 53--64~cM on Chromosome 8.
The univariate QTL results for this region are shown in
Figure~\ref{fig:chr8-lod}.

\begin{figure}
\includegraphics[width = \textwidth]{../qtl2pleio-manuscript/Rmd/chr8-lods.eps}
\caption{Chromosome 8 univariate LOD scores for percent time in light
  and hot plate latency reveal broad, overlapping peaks between 53 cM
  and 64 cM. The peak for percent time in light spans the region from
  approximately 53 cM to 60 cM, with a maximum near 55 cM. The peak
  for hot plate latency begins near 56 cM and ends about 64 cM.}
\label{fig:chr8-lod}
\end{figure}


The estimated QTL allele effects for the two traits are quite
different (Figure~\ref{fig:chr8-effects}).
With the QTL placed at 55~cM, for ``percent time in light'', the WSB and PWK alleles are associated
with large phenotypes and NOD with low phenotypes.
For ``hot plate latency'', on the other hand,
CAST and NZO show low phenotypes and NOD and PWK are near the center.

\begin{figure}
\includegraphics[width = \textwidth]{../qtl2pleio-manuscript/Rmd/coefs.eps}
\caption{Chromosome 8 univariate LOD scores for percent time in light
  and hot plate latency reveal broad, overlapping peaks between 53 cM
  and 64 cM. The peak for percent time in light spans the region from
  approximately 53 cM to 60 cM, with a maximum near 55 cM. The peak
  for hot plate latency begins near 56 cM and ends about 64 cM.}
\label{fig:chr8-effects}
\end{figure}

In applying our test for pleiotropy,  we performed a two-dimensional, two-QTL scan for the pair of
phenotypes. With these results, we created a profile LOD plot
(Figure~\ref{fig:profiles}). The profile LOD for ``percent
time in light'' (in brown) peaks near 55 cM, as was seen in the univariate
analysis.  The profile LOD for ``hot plate latency'' (in blue) peaks near 57 cM,
also similar to the univariate analysis.
The pleiotropy trace (in gray) peaks near 55 cM.

\begin{figure}
\includegraphics[width = \textwidth]{../qtl2pleio-manuscript/Rmd/profile.eps}
\caption{Profile LOD curves for the pleiotropy vs.\ separate QTL
  hypothesis test for ``percent time in light'' and ``hot plate latency''.
  Gray trace denotes pleiotropy LOD values. Triangles denote the
  univariate LOD maxima, while diamonds denote the profile LOD maxima.
  For ``percent time in light'', the brown triangle obscures the
  smaller brown diamond. Likelihood ratio test statistic value
  corresponds to the height of the blue and brown traces at their
  maxima.}
\label{fig:profiles}
\end{figure}

The likelihood ratio test statistic for the test of pleiotropy was
1.2. Based on a parametric bootstrap with 1,000 bootstrap replicates,
the estimated p-value was 0.11, indicating weak
evidence for distinct QTL for the two traits.









\section{Discussion}

We developed a test of pleiotropy vs.\ separate QTL for multiparental
populations, extending the work of \citet{jiang1995multiple} for
multiple alleles and with a linear mixed model to account for
population structure \citep{kang2010variance, yang2014advantages}. Our simulation
studies indicate that the test has power to detect presence of
separate loci, especially when univariate trait associations are
strong (Figure~\ref{fig:power}). Type I error rates indicate that our
test is slightly conservative (Table~\ref{table-typeI}).

In the application of our method to two behavioral phenotypes in a
study of 261 Diversity Outbred mice
\citep{recla2014precise,logan2013high}, we obtained weak evidence
(p=0.11) for the presence of two distinct QTL, with one QTL (which
contained the \textit{Hydin} gene) affecting only ``hot plate latency'' and a
second QTL affecting ``percent time in light'' (Figure~\ref{fig:profiles}).

Founder allele effects plots provide further evidence for the presence
of two distinct loci. As \citet{macdonald2007joint} and
\citet{king2012genetic} have demonstrated in their analyses of multiparental
\emph{Drosophila} populations, a biallelic pleiotropic QTL would result in
allele effects plots that have similar patterns. While we don't know
that ``percent time in light'' and ``hot plate latency'' arise from
biallelic QTL, the dramatic differences that we observe in allele
effects patterns further support the argument for two distinct loci.

We have implemented our methods in an R package
\texttt{qtl2pleio}, but analyses can be computationally intensive and
time consuming. \texttt{qtl2pleio} is written mostly in R, and so we
could likely obtain improved computational speed by porting parts of
the calculations to a compiled language such as C or C++.
To accelerate our multi-dimensional QTL
scans, we have integrated C++ code into \texttt{qtl2pleio},
using the Rcpp package \citep{eddelbuettel2011rcpp}.

Another computational bottleneck is the estimation of the variance
components $V_g$ and $V_e$. To accelerate this procedure, 
especially for the joint analysis of more than two traits, we will
consider other strategies for variance component estimation, including
that described by \citet{hannah2018limmbo}. \citet{hannah2018limmbo}, in joint analysis of dozens of traits, implement a bootstrap
strategy to estimate variance components for lower-dimensional
phenotypes before combining bootstrap estimates
into valid covariance matrices for the full multivariate phenotype. 
Such an approach may ease some of the computational burdens that we encountered.


We view tests of pleiotropy as complementary to 
mediation tests and related methods that have become popular for
inferring biomolecular causal relationships
\citep{chick2016defining,schadt2005integrative,baron1986moderator}. A
mediation test proceeds by including a putative mediator as a
covariate in the regression analysis of phenotype and QTL genotype;
a substantial reduction in the association between
genotype and phenotype corresponds to evidence of mediation. 


Mediation analyses and our pleiotropy test ask distinct, but related, questions. Mediation analysis seeks to establish causal relationships among traits, including molecular traits, or dependent biological and behavioral processes. Pleiotropy tests examine whether two traits share a single source of genetic variation, which may act in parallel or in a causal network. Pleiotropy is required for causal relations among traits. In many cases, the pleiotropy hypothesis is the only reasonable one. 

\citet{schadt2005integrative} argued that
both pleiotropy tests and causal inference methods may contribute to gene network
reconstruction. They developed a model selection strategy, based on
the Akaike Information Criterion \citep{akaike1974new}, to determine which
causal model is most compatible with the observed data.
\citet{schadt2005integrative} extended the methods of
\citet{jiang1995multiple} to consider more complicated alternative
hypotheses, such as the possibility of two QTL, one of which
associates with both traits, and one of which associates with only one
trait. As envisioned by \citet{schadt2005integrative}, we foresee
complementary roles emerging for our pleiotropy test
and mediation tests in the dissection of complex trait genetic
architecture.

CAPE (Combinatorial Analysis of Pleiotropy and Epistasis) is a strategy for identifying higher-order relationships among traits and marker genotypes \citep{tyler2013cape}. 
\citet{tyler2017epistatic} used CAPE to identify epistatic gene networks in Diversity Outbred mice. \citet{tyler2016weak} found evidence for weak epistasis in a large intercross population. 

CAPE uses linear models that are distinct from those in our pleiotropy test. 
A CAPE starts with founder allele dosages at all markers and a collection of two or more traits \citep{tyler2017epistatic}. 
After eigendecomposition to get two or more eigentraits, univariate QTL scans are performed and founder allele effects are estimated at all markers (or a subset of all markers). 
Next, one identifies markers with sufficiently strong effects of at least one founder allele. 
Resulting (eigentrait, marker, founder allele) triples are then subjected to a second model fitting. 
This second round of modeling involves two (eigentrait, marker, founder allele) triples that share an eigentrait. 
The shared (univariate) eigentrait is modeled as a linear function of the two founder allele dosages and their interaction. 

In comparing CAPE and our pleiotropy test, it's important to recognize that the two methods ask different questions. 
CAPE enables assessment of interactions among specific founder allele dosages at two (possibly identical) markers while examining one eigentrait at a time. 
Our pleiotropy test, on the other hand, jointly models two phenotypes and quantifies the evidence against the pleiotropy hypothesis by performing a two-dimensional QTL scan over a genomic region. 
One limitation of CAPE is its inability to account for population structure. Incorporation of a polygenic random effect into CAPE's linear models may improve its performance in multiparental populations. 
CAPE's methods also highlight a current limitation of our pleiotropy test. 
While our multivariate linear models accommodate interactions between founder allele dosages at two markers, we haven't yet incorporated this functionality into our software. 
It is one direction for future research.



Technological advances
in mass spectrometry and RNA sequencing have enabled the acquisition of
high-dimensional biomolecular phenotypes
\citep{ozsolak2011rna,han2012multi}. Multiparental populations in
\textit{Arabidopsis}, maize, wheat, oil palm, rice,
\textit{Drosophila}, yeast, and other organisms enable high-precision
QTL mapping \citep{yu2008genetic, tisne2017identification,
  stanley2017genetic, raghavan2017approaches, mackay2012drosophila,
  kover2009multiparent, cubillos2013high}. The need to analyze
high-dimensional phenotypes in multiparental populations compels the
scientific community to develop tools to study genotype-phenotype
relationships and complex trait architecture. Our test, and its future
extensions, will contribute to these ongoing efforts.




\subsection*{Acknowledgments}

The authors thank Lindsay Traeger, Julia Kemis, and Rene Welch for
valuable suggestions to improve the manuscript. This work was
supported in part by National Institutes of Health grant R01GM070683
(to K.W.B.). The research made use of compute resources and assistance
of the UW-Madison Center For High Throughput Computing (CHTC) in the
Department of Computer Sciences at UW-Madison, which is supported by
the Advanced Computing Initiative, the Wisconsin Alumni Research
Foundation, the Wisconsin Institutes for Discovery, and the National
Science Foundation, and is an active member of the Open Science Grid,
which is supported by the National Science Foundation and the U.S.
Department of Energy's Office of Science.





%\bibliography{research,qtl2pleio-manuscript}

%\newpage
%\appendix

% supplemental tables
%\renewcommand{\thetable}{\textbf{S\arabic{table}}}
%\setcounter{table}{0}


%\begin{table}
%  \caption{Eight founder lines and their one-letter abbreviations.}
%  \label{table-letters}
%\begin{center}
%\small
%  \begin{tabular}{ c | c }
%    \hline
%    Founder allele & One-letter abbreviation \\ \hline
%    A/J & A \\
%    C57BL/6J & B \\
%    129S1/SvImJ & C \\
%    NOD/ShiLtJ & D\\
%    NZO/H1LTJ & E\\
%    Cast/EiJ & F\\
%    PWK/PhJ & G\\
%    WSB/EiJ & H\\
%    \hline
%  \end{tabular}

%\end{center}
%  \end{table}

%\clearpage

  % latex table generated in R 3.5.1 by xtable 1.8-3 package
% Tue Nov 20 10:28:47 2018
%\begin{table}
%\caption{Both ``hot plate latency'' and ``percent time in light''
%  demonstrate multiple QTL peaks with LOD scores above 5.}
%  \label{table-peaks}
%\begin{center}
%\begin{tabular}{l|lrr}
%  \hline
%phenotype & chr & pos & LOD score \\
%   \hline
%percent time in light & 8 & 55.28 & 5.27 \\
% hot plate latency & 8 & 57.77 & 6.22 \\
% percent time in light & 9 & 36.70 & 5.42 \\
% hot plate latency & 9 & 46.85 & 5.22 \\
% percent time in light & 11 & 63.39 & 6.46 \\
% hot plate latency & 12 & 43.52 & 5.13 \\
% percent time in light & 15 & 15.24 & 5.67 \\
% hot plate latency & 19 & 47.80 & 5.48 \\
%   \hline
%\end{tabular}
%\end{center}
%\end{table}%







%\clearpage


% supplemental figures
%\renewcommand{\thefigure}{\textbf{S\arabic{figure}}}
%\setcounter{figure}{0}

%\begin{figure}
%\includegraphics[width = \textwidth]{../qtl2pleio-manuscript/Rmd/scatter.eps}
%\caption{Scatter plot of ``hot plate latency'' against ``percent time in
%  light'', after applying logarithm transformations and winsorizing
%  both traits.}
%\label{fig:scatter}
%\end{figure}


%\begin{figure}
%\includegraphics{../qtl2pleio-manuscript/Rmd/genomewide_lods_10-22.eps}
%\caption{Genome-wide QTL scan for percent time in light reveals
%  multiple QTL, including one on Chromosome 8.}
%\label{fig:genomewide10-22}
%\end{figure}



%%% Chapter 2 end


\chapter{Applications}
%%%%% 3A start
\section{Expression trait hotspot dissection}
\subsection{Introduction}

A central goal of systems genetics studies is to identify causal relationships between biomolecules. Recent work by \citet{chick2016defining}, which builds on research from \citet{baron1986moderator}, has popularized linear regression-based methods for causal inference in genetics. This development motivates our goal of clarifying a role for our pleiotropy test. We argue that our pleiotropy test complements mediation analysis in two ways. First, our test limits the set of candidate mediators by ruling out traits that don't share a single pleiotropic QTL. Second, when regression-based mediation analysis fails to identify a mediator, our pleiotropy test still provides information on the number of QTL, which may provide clues to aid biological understanding.


We begin by discussing what it means for one trait to mediate a relationship between a DNA variant and another trait. To clarify our discussion, we refer to an example from \citet{chick2016defining} (Figure~\ref{fig:Dhtkd1}). \citet{chick2016defining}, in studying livers of 192 Diversity Outbred mice, found evidence that \emph{Dhtkd1} transcript levels associated with a local marker on Chromosome 2. They also found that the same marker affected DHTKD1 protein concentrations. 
As anticipated, mediation analysis, in which the DHTKD1 protein concentrations are regressed on founder allele dosages (at the Chromosome 2 marker) demonstrated that \emph{Dhtkd1} transcript levels act as a mediator between DHTKD1 protein concentrations and the founder allele dosages. In fact, the extent of the reduction in association strength indicates that the primary pathway by which the genetic marker affects DHTKD1 protein concentrations is through \emph{Dhtkd1} transcript levels.

\begin{figure}
  \centering
  \includegraphics[width = 0.7\textwidth]{../QTLfigs/Figs/central_dogma-Dhtkd1.pdf}
  \caption{A DNA variant in the \emph{Dhtkd1} gene affects \emph{Dhtkd1} transcript abundances which, in turn, affect DHTKD1 protein concentrations.}\label{fig:Dhtkd1}
\end{figure}




\citet{crick1958protein} articulated a pathway for transmission of biological information that is now known as the central dogma of molecular biology (Figure~\ref{fig:dogma}). In it, he argued that information encoded in DNA sequence is transmitted via transcription to RNA molecules, which, in turn, transfer the information to proteins via translation. This sequence of information transfer, from DNA to RNA to protein, provides a natural setting by which to examine mediation analysis. If a DNA variant affects protein concentrations only through its RNA transcripts, then conditioning on RNA transcript levels would greatly reduce the strength of association between DNA variant and protein concentration. In the example above, a DNA variant in the \emph{Dhtkd1} gene affects its transcript levels, which then affect DHTKD1 protein levels (Figure~\ref{fig:Dhtkd1}).


\begin{figure}
  \centering
  \includegraphics[width = 0.7\textwidth]{../QTLfigs/Figs/central_dogma.pdf}
  \caption{Biological information is encoded in DNA. This information is passed, via transcription, to RNA molecules. The process of translation transmits the information to proteins.}\label{fig:dogma}
\end{figure}




Following \citet{keller2018genetic}, we generalize this setting to the case where a DNA variant affects a local transcript level, which then affects a nonlocal transcript level. We term a transcript ``local'' to a marker when its DNA is near that marker. Typically, we use a threshold of no more than 2 Mb to restrict the number of local transcripts for a given marker. A nonlocal transcript, then, is either one that arises from a gene on another chromosome or from a distant gene on the same chromosome as the marker.

It is highly plausible that concentration variations in one transcript may affect abundances of a second transcript. For example, the first transcript may encode a transcription factor protein. In this case, the transcription factor protein may influence expression patterns of the second transcript (and perhaps other transcripts). Alternatively, the first transcript may act directly to enhance expression of a second transcript, possibly by interacting with a complex of biomolecules in the nucleus.





To determine whether a local transcript level mediates the relationship between a nonlocal transcript level and a DNA variant, we perform a series of regression analyses, which we detail below (Equations \ref{model1}, \ref{model2}, \ref{model3}, \ref{model4}). In brief, we regress the nonlocal transcript levels on founder allele dosages at the DNA variant, with and without conditioning on the candidate mediator (the local transcript levels). If the LOD score diminishes sufficiently upon conditioning on a candidate, then we declare the candidate a mediator.

The rationale behind this strategy is as follows. If the DNA variant affects the nonlocal transcript levels solely by way of local transcript levels, then conditioning on the local transcript levels would nullify the relationship between DNA variant and the nonlocal transcript levels. At the other extreme, if the DNA variant affects nonlocal transcript levels solely through mechanisms that don't involve the local transcript levels, then conditioning on local transcript levels would not affect the association between the DNA variant and nonlocal transcript levels.

One role for our pleiotropy test is to potentially limit the set of candidate mediators that require study. For example, if local transcript levels and nonlocal transcript levels map to separate QTL, then it would be unlikely that local transcript levels affect the nonlocal transcript levels (or vice versa). On the other hand, if we believe that a single pleiotropic locus affects both local transcript levels and nonlocal transcript levels, then a mediation analysis is particularly useful in efforts to clarify the relationship between the local transcript levels and nonlocal transcript levels. In this setting, where a single pleiotropic locus affects both local transcipt levels and nonlocal transcript levels, and we seek to identify the intermediate, we would perform two mediation analyses, one with local transcript levels as candidate intermediate between DNA variant and nonlocal transcript levels, and a second analysis with nonlocal transcript levels as a candidate intermediate between DNA variant and local transcript levels.

%\subsection{Regression-based mediation methods}

%Frequently one uses regression-based methods to identify causal intermediates from among multiple measured variables \citep{baron1986moderator}. For example, \citet{chick2016defining} measured 16,921 gene transcript levels and 6,756 protein concentrations in liver tissue from 192 mice. After identifying QTL for gene transcript levels and protein concentrations, they performed regression-based mediation analyses. For each candidate mediator, they fitted four linear models:

\begin{frameenv}{Four regressions for a single mediation analysis}\label{frame3}


%\begin{equ}[!ht]
\begin{equation}
Y = b1 + WC + E
\label{model1}
\end{equation}
%\caption{Linear model with intercept and covariates only.}
%\end{equ}

%\begin{equ}[!ht]
\begin{equation}
Y = XB + WC + E
\label{model2}
\end{equation}
%\caption{Linear model with founder allele dosages and covariates.}
%\end{equ}

%\begin{equ}[!ht]
\begin{equation}
Y = b1 + WC + M\beta + E
\label{model3}
\end{equation}
%\caption{Linear model with intercept, covariates, and candidate mediator.}
%\end{equ}

%\begin{equ}[!ht]
\begin{equation}
Y = XB + WC + M\beta + E
\label{model4}
\end{equation}
%\caption{Linear model with founder allele dosages, covariates, and candidate mediator.}
%\end{equ}
\end{frameenv}

In the four linear regression models (Frame~\ref{frame3}), $X$ is a $n$ by $8$ matrix of founder allele dosages at a single marker, $B$ is a $8$ by $1$ matrix of founder allele effects, $E$ is a $n$ by $1$ matrix of random errors, $b$ is an number, $1$ is a $n$ by $1$ matrix with all entries set to $1$, $Y$ is a $n$ by $1$ matrix of phenotype values (for a single trait), and $M$ is a $n$ by $1$ matrix of values for a putative mediator. $C$ is a matrix of covariate effects, and $W$ is a matrix of covariates. We denote the coefficient of the mediator by $\beta$.

We assume that the vector $E$ is (multivariate) normally distributed with zero vector as mean and covariance matrix $\Sigma = \sigma^2I_n$, where $I_n$ is the $n$ by $n$ identity matrix.

With the above models, the log-likelihoods are easily calculated. For example, in Equation \ref{model1}, the vector $Y$ follows a multivariate normal distribution with mean $(b1 + WC)$ and covariance $\Sigma = \sigma^2I$. Thus, we can write the likelihood for Model \ref{model1} as:

\begin{equation}
    L(b, C, \sigma^2| Y, W) = (2\pi)^{- \frac{n}{2}}\exp{ \left(- \frac{1}{2}(Y - b1 - WC)^T\Sigma^{-1}(Y - b1 - WC)\right)}
\end{equation}

We thus have the following equation (~\ref{eq:ll} for the log-likelihood for Model~\ref{model1}:

\begin{equation}
    \log L(b, C, \sigma^2 | Y, W) = - \frac{n}{2}\log (2\pi) - \frac{1}{2} (Y - b1 - WC)^T\Sigma^{-1}(Y - b1 - WC)\label{eq:ll}
\end{equation}


\citet{chick2016defining} calculated the log$_{10}$ likelihoods for all four models before determining two LOD scores:
\begin{equation}
LOD_1 = log_{10}(\text{Model~\ref{model2} likelihood}) - log_{10}(\text{Model~\ref{model1} likelihood})\label{eq:lod1}
\end{equation}

\begin{equation}
LOD_2 = log_{10}(\text{Model~\ref{model4} likelihood}) - log_{10}(\text{Model~\ref{model3} likelihood})\label{eq:lod2}
\end{equation}

And, finally, \citet{chick2016defining} calculated the statistic:

\begin{equation}
\text{LOD difference} = LOD_1 - LOD_2\label{eq:lod-diff}
\end{equation}

LOD difference need not be positive. For example, in the setting where the putative mediator is not a true mediator, $LOD_1$ and $LOD_2$ values may lead to a negative LOD difference statistic. In our analyses in this section, we truncated these negative LOD difference values to zero.

For each mediation analysis, we consider the LOD difference proportion, which we define below (Equation \ref{eq:lod-diff-prop}).

\begin{equation}
\text{LOD difference proportion} = \frac{(LOD_1 - LOD_2)}{LOD_1}
\label{eq:lod-diff-prop}
\end{equation}

In other words, we consider what proportion of the association strength, on the LOD scale, is diminished by conditioning on a putative mediator. This statistic differs from the LOD difference statistic in that it scales the LOD difference by LOD$_1$ value. We do this in efforts to accommodate the diversity of LODs in our data. While a value of 5 for LOD difference may be important for a trait with a LOD$_1$ of 12, another trait with a LOD$_1$ of 120 that has a LOD$_2$ of 115, suggests that nearly all of the signal remains transmitted when conditioning on the putative mediator.







%\subsection{Four assumptions for causal inference}

To claim that the LOD difference statistic reflects a causal relationship, four assumptions about confounding are needed \citep{vanderweele2015explanation}.


%\begin{mdframed}
%\caption{XXXX}
%\begin{enumerate}
%\item No unmeasured confounding of the treatment-outcome %relationship
%\item No unmeasured confounding of the mediator-outcome %relationship
%\item No unmeasured confounding of the treatment-mediator relationship
%\item No mediator-outcome confounder that is affected by the treatment
%\label{list4}
%\end{enumerate}
%\end{mdframed}

\begin{frameenv}{Four assumptions for causal inference}\label{frame1}
  \begin{enumerate}
\item No unmeasured confounding of the DNA variant-nonlocal transcript levels relationship
\item No unmeasured confounding of the local transcript levels-nonlocal transcript levels relationship
\item No unmeasured confounding of the DNA variant-local transcript levels relationship
\item No local transcript levels-nonlocal transcript levels confounder that is affected by the DNA variant
\end{enumerate}

\end{frameenv}






It is often difficult to recognize unmeasured confounders. In studies of Diversity Outbred mice, relatedness is a possible confounder, yet the linear models (Equations~\ref{model1},~\ref{model2},~\ref{model3},~\ref{model4}) fail to account for the complex relatedness patterns among Diversity Outbred mice. Scientists have developed a suite of sensitivity analysis tools to characterize possible unmeasured confounders and their possible impact in mediation analysis. While a discussion of sensitivity analysis is beyond the scope of this thesis, it may be useful in future studies to assess robustness of mediation analysis results in the presence of unmeasured confounders. \citet{vanderweele2015explanation} discusses sensitivity analysis in detail. 







%\subsection{Declaring mediators}

Multiple research teams have proposed strategies for assessing statistical significance of the ``LOD difference'' statistic. \citet{chick2016defining} used individual transcript levels as sham mediators and tabulated their LOD difference statistics. They then compared the observed LOD difference statistics for putative mediators to the empirical distribution of LOD difference statistics obtained from the collection of sham mediators. \citet{keller2018genetic}, on the other hand, in their study of pancreatic islet cell biology, declared mediators those local transcripts that diminished the LOD score of nonlocal transcripts by at least 1.5 \citep{keller2018genetic}. While significance threshold determination remains an active area of research, we proceed below by examining all 147 nonlocal transcript levels that \citet{keller2018genetic} identified as mapping to the Chromosome 2 hotspot.



\subsection{Methods}

We examined the potential that the two methods - 1. pleiotropy vs. separate QTL testing and 2. mediation analysis - could play complementary roles in efforts to dissect gene expression trait hotspots. After describing our data below, we explain our statistical analyses involving 13 local gene expression traits and 147 nonlocal gene expression traits, all of which map to a 4-Mb hotspot on Chromosome 2.

%\subsection{Data description}

We analyzed data from 378 Diversity Outbred mice \citep{keller2018genetic}. \citet{keller2018genetic} genotyped tail biopsies with the GigaMUGA microarray \citep{morgan2015mouse}. They also used RNA sequencing to measure genome-wide pancreatic islet cell gene expression for each mouse at the time of sacrifice \citep{keller2018genetic}. They shared these data, together with inferred founder allele probabilities, on the Data Dryad site (\url{https://datadryad.org/resource/doi:10.5061/dryad.pj105}).

We examined the Chromosome 2 pancreatic islet cell expression trait hotspot that \citet{keller2018genetic} identified. \citet{keller2018genetic} found that 147 nonlocal traits map to the 4-Mb region centered at 165.5 Mb on Chromosome 2. The 147 nonlocal traits all exceeded the genome-wide LOD significance threshold, 7.18 \citep{keller2018genetic}. With regression-based mediation analyses, they identified transcript levels of local gene \emph{Hnf4a} as a mediator of 88 of these 147 nonlocal traits.

Because \citet{keller2018genetic} reported that some nonlocal traits that map to the Chromosome 2 hotspot did not demonstrate evidence of mediation by \emph{Hnf4a} expression levels, we elected to study a collection of local gene expression traits, rather than \emph{Hnf4a} alone. This strategy enabled us to ask whether one of twelve other local traits mediates those nonlocal hotspot traits that are not mediated by \emph{Hnf4a}. Our set of local gene expression traits includes \emph{Hnf4a} and 12 other local genes (Table \ref{tab:annot}). Our 13 local genes are the only genes that met all of the following criteria:

\begin{frameenv}{Local gene inclusion criteria}\label{frame2}

\begin{enumerate}
    \item QTL peak with LOD $>$ 40
    \item QTL peak position within 2 Mb of hotspot center (165.5 Mb)
    \item Gene midpoint within 2 Mb of hotspot center (165.5 Mb)
\end{enumerate}
\end{frameenv}

The 147 nonlocal traits that we studied all had LOD peak heights above 7.18 and LOD peak positions within 2 Mb of the center of the hotspot (at 165.5 Mb).



%\subsection{Statistical analyses}

We used both univariate and bivariate QTL scans. In the univariate scans, we identified LOD peaks for each of 160 (147 nonlocal and 13 local) expression traits that mapped to within 2 Mb of the center of the Chromosome 2 expression trait hotspot. \citet{keller2018genetic} provided univariate QTL mapping results in their Data Dryad file, so we didn't repeat the univariate QTL scans. For a given univariate phenotype, \citet{keller2018genetic} fitted a linear mixed effects model (with polygenic random effect) at every marker across the genome. \citet{keller2018genetic} performed calculations with the \texttt{qtl2} R package \citep{broman2019rqtl2}.



%\subsubsection{Bivariate QTL scans}

Our analyses involved 13 local expression traits and 147 nonlocal expression traits. We described above (Frame~\ref{frame2}) the criteria for choosing these expression traits.

We performed a series of two-dimensional QTL scans in which we paired each local gene's transcript levels with each nonlocal gene's transcript levels, for a total of 13 * 147 = 1,911 two-dimensional scans. Each scan examined the same set of 180 markers, which spanned the interval from 163.1 Mb to 167.8 Mb and included univariate peaks for all 13 + 147 = 160 expression traits. We performed these analyses with the R package \texttt{qtl2pleio} \citep{qtl2pleio}.

For each QTL scan, we fitted a collection of bivariate models for all 180 * 180 = 32,400 ordered pairs of markers. Throughout our analyses, we treated the founder allele dosages as known, despite the fact that they are inferred through hidden Markov model methods \citep{broman2012genotype, broman2012haplotype, broman2019rqtl2}.

For each ordered pair of markers, we fitted a multivariate linear mixed effects model with the methods of Chapter 2.

%\subsubsection{Mediation analyses}

We performed linear regression-based mediation analyses for all 1,911 local-nonlocal trait pairs. We probed the extent to which each nonlocal trait's association strength diminished upon conditioning on transcript levels of a putative mediator. Each of the 13 local expression traits, considered one at a time, served as putative mediators. We thus fitted the four linear regression models that we describe above (Equations \ref{model1}, \ref{model2}, \ref{model3}, \ref{model4}).

One question that needs clarification is the choice of ``driver'' for each mediation analysis. We elected to use the founder allele probabilities at the marker that demonstrated the univariate LOD peak for the putative mediator. Alternative analyses, in which one chooses as driver the founder allele probabilities at which the nonlocal trait has its univariate peak, are also possible.



%\subsubsection{Local gene analyses}

To visualize the summary statistics for our two methods, we plotted, for each local gene, a scatterplot of LOD difference proportion values against pleiotropy vs. separate QTL test statistics.

% latex table generated in R 3.5.1 by xtable 1.8-3 package
% Mon Dec 17 10:04:32 2018
\begin{table}[ht]
\centering
\begin{tabular}{lrrrr}
  \hline
gene & start & end & QTL peak position & LOD\\
  \hline
Pkig & 163.66 & 163.73 & 163.52 & 51.68 \\
  Serinc3 & 163.62 & 163.65 & 163.58 & 126.93 \\
  Hnf4a & 163.51 & 163.57 & 164.02 & 48.98 \\
  Stk4 & 164.07 & 164.16 & 164.03 & 60.39 \\
  Pabpc1l & 164.03 & 164.05 & 164.03 & 52.50 \\
  Slpi & 164.35 & 164.39 & 164.61 & 40.50 \\
  Neurl2 & 164.83 & 164.83 & 164.64 & 64.58 \\
  Cdh22 & 165.11 & 165.23 & 165.05 & 53.84 \\
  2810408M09Rik & 165.49 & 165.49 & 165.57 & 67.34 \\
  Eya2 & 165.60 & 165.77 & 165.72 & 98.89 \\
  Prex1 & 166.57 & 166.71 & 166.75 & 46.91 \\
  Ptgis & 167.19 & 167.24 & 167.27 & 56.25 \\
  Gm14291 & 167.20 & 167.20 & 167.27 & 73.72 \\
   \hline
\end{tabular}
\caption{Local gene annotations for analysis of Chromosome 2 expression trait hotspot. All positions are in units of Mb on Chromosome 2. LOD peak position and LOD peak height refer to those obtained from univariate analyses. ``Start'' and ``end'' refer to the local gene's DNA start and end positions, as annotated by Ensembl version 75.}
\label{tab:annot}
\end{table}

%\subsubsection{Nonlocal gene analyses}

We also examined the 1,911 pairs from the per-nonlocal gene perspective. For each of the 147 nonlocal genes, we plotted LOD difference proportion against pleiotropy test statistic values. We present below (Figure \ref{fig:4nonlocal}) examples that illustrate some of the observed patterns between pleiotropy test statistics and LOD difference proportion values.


\subsection{Results}

%\subsection{Scatter plot for all 1911 pairs}

We present below our scatter plot for all 147 * 13 = 1,911 pairs of traits (Figure \ref{fig:lod-diff-prop-v-lrt-all}). Each pair contains one local expression trait and one nonlocal expression trait. Each point in the figure represents a single pair. We see that points with high values of LOD difference proportion tend to have small values of pleiotropy test statistic, and those points with high values of the pleiotropy test statistic tend to have small values of LOD difference proportion.

We colored blue the points that involve \emph{Hnf4a} as the local gene; all other local genes have red points. The most striking feature of the coloring is that many blue points have small values of the pleiotropy test statistics and very high values of LOD difference proportion.



\begin{figure}
    \centering
    \includegraphics[width = 0.7\textwidth]{../keller2018-chr2-hotspot-chtc/Rmd/lod-diff-prop-v-lrt.eps}
    \caption[LOD difference proportion vs. pleiotropy test statistic]{LOD difference proportion against pleiotropy vs. separate QTL test statistics for 1911 pairs of traits. Each point represents a single pair. Pairs that involve local expression trait Hnf4a are colored blue, while all others are colored red. Note the high prevalence of blue points in the upper left quadrant of the figure. These points, with low values of the pleiotropy vs. separate QTL test statistic and high values of LOD difference proportion, are consistent with \emph{Hnf4a} transcript levels mediating the effect of \emph{Hnf4a} genetic variants affecting nonlocal transcript abundances.}
    \label{fig:lod-diff-prop-v-lrt-all}
\end{figure}

%\subsection{Local gene analyses}

To more thoroughly examine the relationships across the 13 local genes, we created 13 plots of LOD difference proportion (from the mediation analyses) against pleiotropy test statistic. They reveal common patterns. First, we see no points in the upper right quadrant of each plot. This tells us that those nonlocal genes with high values of pleiotropy test statistic have low values of LOD difference proportion. Similarly, those nonlocal genes with high values of LOD difference proportion tend to have small values of the pleiotropy test statistic. Finally, some trait pairs demonstrate low values of both the LOD difference proportion and pleiotropy test statistic. This observation suggests that, for a given local expression trait, these pairs are not mediated by the local expression trait yet they arise from a shared pleiotropic locus.

In comparing the \emph{Hnf4a} plot (Figure \ref{fig:hnf4a}) with the other 12 plots (Figure \ref{fig:nothnf4a-12}), we see that none of the 12 plots in Figure \ref{fig:nothnf4a-12} closely resembles Figure \ref{fig:hnf4a}. \emph{Serinc3}, \emph{Stk4}, \emph{Neurl2}, and \emph{Cdh22} are closest in appearance to the plot of \emph{Hnf4a}. However, each of \emph{Serinc3}, \emph{Stk4}, \emph{Neurl2}, and \emph{Cdh22} has very few points with LOD difference proportion above 0.5, while \emph{Hnf4a} has many points with LOD difference proportion above 0.5.

\begin{figure}
    \centering
    \includegraphics[width = \textwidth]{../keller2018-chr2-hotspot-chtc/Rmd/Hnf4a-lod-diff-prop-v-lrt.eps}
    \caption[LOD difference proportion vs. pleiotropy test statistic.]{Scatter plot of LOD difference proportion against pleiotropy vs. separate QTL test statistic for 147 pairs of traits. Each pair includes \emph{Hnf4a} and one of the nonlocal gene expression traits that map to the Chromosome 2 hotspot.}
    \label{fig:hnf4a}
\end{figure}

\begin{figure}
    \centering
    \includegraphics[width = 0.8\textwidth]{../keller2018-chr2-hotspot-chtc/Rmd/12local-facet_grid-no-strip.eps}
    \caption[LOD difference proportion vs. pleiotropy test statistic from the per-local gene perspective.]{Scatter plots of LOD difference proportion against pleiotropy vs. separate QTL test statistic with 147 pairs of traits per panel. Each panel includes a local gene expression trait (per the label in the upper right quadrant) and one of the 147 nonlocal gene expression traits that map to the Chromosome 2 hotspot.}
    \label{fig:nothnf4a-12}
\end{figure}






% latex table generated in R 3.5.1 by xtable 1.8-3 package
% Wed Dec 19 10:50:32 2018
\begin{table}[ht]
\centering
\begin{tabular}{lc}
  \hline
 Local gene & Number of nonlocal gene expression traits \\
  \hline
2810408M09Rik &   3 \\
  Cdh22 &   4 \\
  Eya2 &   4 \\
  Gm14291 &   5 \\
  Hnf4a &  89 \\
  Neurl2 &  15 \\
  Pabpc1l &   4 \\
  Pkig &   2 \\
  Prex1 &   5 \\
  Ptgis &   3 \\
  Serinc3 &   7 \\
  Slpi &   1 \\
  Stk4 &   5 \\
   \hline
\end{tabular}
\caption{Number of nonlocal gene expression traits for which each local gene expression trait is the strongest mediator.}
\label{tab:med-count}
\end{table}

%\subsection{Nonlocal gene analyses}

Scatter plots of LOD difference proportion values against pleiotropy vs. separate QTL test statistics for each of the 147 nonlocal expression traits demonstrated multiple patterns. Eighty-nine nonlocal genes' plots showed \emph{Hnf4a} to have the greatest value, among the 13 putative mediators, of the LOD difference proportion statistic. In many cases, \emph{Hnf4a}'s LOD difference proportion statistic was at least twice that of any of the other 12 local gene expression levels.

\begin{figure}
    \centering
    \caption{Scatter plots for four nonlocal expression traits. Each plot features 13 points, one for each local gene expression trait. The vertical axis denotes LOD difference proportion values, while the horizontal axis corresponds to pleiotropy test statistics. Blue points represent the pairing with local gene expression trait \emph{Hnf4a}. Red points represent the other 12 local gene expression traits.}
    \begin{subfigure}[t]{.45\textwidth}
        \includegraphics[width = \textwidth]{../keller2018-chr2-hotspot-chtc/Rmd/nonlocal-scatter_97.eps}
        \caption[\emph{Hnf4a} transcript levels mediate \emph{Abcb4} transcript levels.]{\emph{Hnf4a} transcript levels mediate \emph{Abcb4} transcript levels. The high LOD difference proportion and the very small pleiotropy test statistic together provide evidence that \emph{Hnf4a} mediates \emph{Abcb4} transcript levels.}
    \end{subfigure}
    \begin{subfigure}[t]{.45\textwidth}
        \includegraphics[width = \textwidth]{../keller2018-chr2-hotspot-chtc/Rmd/nonlocal-scatter_105.eps}
        \caption[\emph{Hnf4a} transcript levels mediate \emph{Smim5} transcript levels.]{\emph{Hnf4a} transcript levels mediate \emph{Smim5} transcript levels. The high LOD difference proportion and the very small pleiotropy test statistic together provide evidence that \emph{Hnf4a} mediates \emph{Smim5} transcript levels.}
    \end{subfigure}
    \begin{subfigure}[t]{.45\textwidth}
        \includegraphics[width = \textwidth]{../keller2018-chr2-hotspot-chtc/Rmd/nonlocal-scatter_64.eps}
        \caption[\emph{Tmprww4} transcript levels are not mediated by any of the 13 local gene expression traits.]{\emph{Tmprww4} transcript levels are not mediated by any of the 13 local gene expression traits. However, several local gene expression traits arise from separate QTL, as evidenced by their large (greater than 5) values of the pleiotropy test statistic.}
    \end{subfigure}
    \begin{subfigure}[t]{.45\textwidth}
        \includegraphics[width = \textwidth]{../keller2018-chr2-hotspot-chtc/Rmd/nonlocal-scatter_141.eps}
        \caption[\emph{Gm3095} transcript levels are not mediated by any of the 13 local genes.]{\emph{Gm3095} transcript levels are not mediated by any of the 13 local genes. The tests of pleiotropy demonstrate evidence of separate QTL for some local expression traits when paired with \emph{Gm3095}.}
    \end{subfigure}
    \label{fig:4nonlocal}
\end{figure}





\subsection{Discussion}

Our pairwise analyses with both mediation analyses and tests of pleiotropy vs. separate QTL provide additional evidence for the importance of \emph{Hnf4a} in the biology of the Chromosome 2 hotspot in pancreatic islet cells. Our analyses, and, specifically, the test of pleiotropy vs. separate QTL, may be more useful when studying nonlocal traits that map to a hotspot yet don't show strong evidence of mediation by local expression traits. In such a setting, the test of pleiotropy vs. separate QTL can, at least, provide some information about the genetic architecture at the hotspot. Specifically, our test of pleiotropy vs. separate QTL may inform inferences about the number of underlying QTL in a given expression trait hotspot. Additionally, our test may limit the number of expression traits that are potential intermediates between a QTL and a specified nonlocal expression trait. This relies on the assumption that an intermediate expression trait and a target expression trait presumably share a QTL.

On the other hand, mediation analyses, when they provide evidence for mediation of a nonlocal trait by a local expression trait, are more informative than the test of pleiotropy vs. separate QTL, since the mediation analyses identify precisely the intermediate expression trait.

We recommend using both tests of pleiotropy vs separate QTL and mediation analyses when dissecting an expression trait QTL hotspot. In practice, mediation analyses, which involve a collection of univariate regressions at a small set of pre-specified markers, are less computationally intense than tests of pleiotropy vs. separate QTL, since the latter requires a two-dimensional QTL scan over the region of interest. This two-dimensional QTL scan often involves fitting more than ten thousand bivariate regression models. In light of the computational cost of testing pleiotropy vs separate QTL, future researchers may wish to use it as a follow-up to mediation analyses when examining expression trait hotspots.

Future research may investigate the use of polygenic random effects in the statistical models for mediation analysis. Additional methodological questions include approaches for declaring significant a mediation LOD difference proportion and consideration of other possible measures and scales of extent of mediation. Additionally, future researchers may wish to consider biological models that contain two mediators.

The social and health sciences have witnessed much methods research in  mediation analysis. The field of statistical genetics has not fully adopted these strategies yet, but, given the nature of current and future data, many opportunities exist for translation of approaches from epidemiology to systems genetics. For example, the 2015 text from Vander Weele contains detailed discussions of many methods issues that arose in mediation analyses in epidemiology studies.





% latex table generated in R 3.5.1 by xtable 1.8-3 package
% Tue Dec 18 18:18:00 2018
{\tiny
\begin{longtable}{lrr}
\caption{LOD peak positions and peak heights for 147 expression traits that map to the Chromosome 2 expression trait hotspot. We see that some transcript levels have LOD scores below the 7.18 genome-wide threshold. We believe that this is due to variations in statistical modeling between our analyses and those of \citet{keller2018genetic}.}\\ 
% perhaps inclusion or exclusion of an intercept term.
\hline \\
%\begingroup\tiny

Gene & Peak position & LOD \\
  \hline
Mtfp1 & 163.52 & 17.79 \\
  Slc12a7 & 163.58 & 8.30 \\
  Gpa33 & 163.58 & 26.38 \\
  Tmem25 & 163.58 & 11.26 \\
  Klhl29 & 163.58 & 11.03 \\
  Slc9a3r1 & 163.58 & 8.40 \\
  Kif1c & 163.58 & 11.12 \\
  Gata4 & 163.58 & 8.72 \\
  Ppp2r5b & 163.58 & 7.80 \\
  Vil1 & 163.58 & 26.48 \\
  Aldh4a1 & 163.58 & 9.38 \\
  Cotl1 & 163.58 & 16.25 \\
  Tmprss4 & 163.58 & 18.30 \\
  Fhod3 & 163.58 & 17.62 \\
  Zfp750 & 163.58 & 7.83 \\
  Svop & 163.58 & 10.12 \\
  Abcb4 & 163.58 & 16.59 \\
  Ccnjl & 163.58 & 6.70 \\
  Sephs2 & 163.58 & 27.21 \\
  Pcdh1 & 163.58 & 12.48 \\
  Fat1 & 163.58 & 7.49 \\
  Sox4 & 163.58 & 14.71 \\
  Gm8206 & 163.58 & 11.33 \\
  Gm8492 & 163.58 & 10.47 \\
  Clcn5 & 164.02 & 10.49 \\
  Slc6a8 & 164.02 & 8.35 \\
  Bcmo1 & 164.02 & 47.31 \\
  Arrdc4 & 164.02 & 8.75 \\
  Cacnb3 & 164.02 & 47.06 \\
  Sec14l2 & 164.02 & 12.00 \\
  Pgrmc1 & 164.02 & 15.88 \\
  Baiap2l2 & 164.02 & 20.33 \\
  Recql5 & 164.02 & 33.97 \\
  Cpd & 164.02 & 13.70 \\
  Degs2 & 164.02 & 15.20 \\
  Muc13 & 164.02 & 18.91 \\
  Clic5 & 164.02 & 10.39 \\
  Tff3 & 164.02 & 16.38 \\
  Myo7b & 164.02 & 13.70 \\
  Afg3l2 & 164.02 & 19.12 \\
  Sema4g & 164.02 & 23.39 \\
  Agap2 & 164.02 & 33.99 \\
  Plxna2 & 164.02 & 8.61 \\
  Aldob & 164.02 & 20.99 \\
  Epb4.1l4b & 164.02 & 9.21 \\
  Sel1l3 & 164.02 & 17.92 \\
  Sult1b1 & 164.02 & 17.63 \\
  Hpgds & 164.02 & 31.14 \\
  Ush1c & 164.02 & 21.90 \\
  Calml4 & 164.02 & 12.62 \\
  Fam83b & 164.02 & 16.12 \\
  Myo15b & 164.02 & 71.57 \\
  Inpp5j & 164.02 & 11.57 \\
  Ttyh2 & 164.02 & 15.20 \\
  Cdhr2 & 164.02 & 30.00 \\
  Myrf & 164.02 & 37.55 \\
  Sh3bp4 & 164.02 & 9.04 \\
  Vgf & 164.02 & 15.05 \\
  Grtp1 & 164.02 & 23.88 \\
  B4galnt3 & 164.02 & 20.43 \\
  Gucy2c & 164.02 & 12.02 \\
  Smim5 & 164.02 & 18.20 \\
  Nrip1 & 164.02 & 9.50 \\
  Clrn3 & 164.02 & 20.13 \\
  Acot4 & 164.02 & 12.17 \\
  Hunk & 164.02 & 18.50 \\
  Zbtb16 & 164.02 & 6.69 \\
  Osgin1 & 164.02 & 13.51 \\
  Zfp541 & 164.02 & 25.28 \\
  2610042L04Rik & 164.02 & 8.14 \\
  Gm9429 & 164.02 & 9.13 \\
  Agxt2 & 164.02 & 14.54 \\
  Gm17147 & 164.02 & 26.94 \\
  Gm8281 & 164.02 & 8.40 \\
  Ddx23 & 164.03 & 9.29 \\
  Map2k6 & 164.03 & 18.54 \\
  Npr1 & 164.03 & 9.68 \\
  Hdac6 & 164.03 & 11.28 \\
  Vav3 & 164.03 & 13.51 \\
  Atp11b & 164.03 & 11.77 \\
  Aadat & 164.03 & 8.08 \\
  Tmem19 & 164.03 & 20.60 \\
  Gbp4 & 164.03 & 7.14 \\
  Papola & 164.03 & 12.14 \\
  Als2 & 164.03 & 18.35 \\
  Sult1d1 & 164.03 & 11.23 \\
  Myo6 & 164.03 & 10.27 \\
  Dnajc22 & 164.03 & 13.72 \\
  Unc5d & 164.03 & 6.96 \\
  Gm3095 & 164.03 & 10.90 \\
  Gm26886 & 164.03 & 8.76 \\
  Eps8 & 164.06 & 12.49 \\
  Ddc & 164.06 & 13.61 \\
  Fras1 & 164.06 & 10.06 \\
  Card11 & 164.06 & 7.20 \\
  Glyat & 164.06 & 10.79 \\
  Pipox & 164.08 & 15.90 \\
  Iyd & 164.08 & 23.23 \\
  Man1a & 164.26 & 8.77 \\
  Cdc42ep4 & 164.26 & 8.11 \\
  Ppara & 164.26 & 8.84 \\
  Galr1 & 164.26 & 10.10 \\
  Ctdsp1 & 164.26 & 7.13 \\
  Eif2ak3 & 164.26 & 8.93 \\
  Misp & 164.26 & 9.24 \\
  Sun1 & 164.26 & 7.35 \\
  Kctd8 & 164.26 & 9.76 \\
  Dcaf12l1 & 164.26 & 8.95 \\
  4930539E08Rik & 164.26 & 8.25 \\
  Slc29a4 & 164.26 & 7.94 \\
  Gm3239 & 164.26 & 9.21 \\
  Gm3629 & 164.26 & 9.77 \\
  Gm3252 & 164.26 & 9.87 \\
  Gm3002 & 164.26 & 7.62 \\
  Ctsh & 164.29 & 9.76 \\
  Dao & 164.29 & 7.06 \\
  Ak7 & 164.31 & 10.07 \\
  Pcp4l1 & 164.35 & 9.11 \\
  Gm12929 & 164.62 & 37.80 \\
  Mgat1 & 164.63 & 10.10 \\
  Atg7 & 164.76 & 7.82 \\
  Gm11549 & 165.05 & 10.23 \\
  Ccdc111 & 165.12 & 7.77 \\
  Fam20a & 165.15 & 7.59 \\
  Oscp1 & 165.16 & 7.99 \\
  Dsc2 & 165.28 & 8.06 \\
  Adam10 & 165.28 & 9.73 \\
  Plb1 & 165.28 & 8.85 \\
  Ccdc89 & 165.28 & 7.84 \\
  9930013L23Rik & 165.28 & 7.93 \\
  Gm13648 & 165.28 & 7.08 \\
  Igfbp4 & 165.45 & 8.68 \\
  Ldlrap1 & 165.45 & 7.18 \\
  Cdh18 & 165.45 & 7.96 \\
  Arhgef10l & 165.46 & 7.57 \\
  Cib3 & 165.48 & 8.70 \\
  Macf1 & 165.57 & 9.10 \\
  Fam63b & 165.58 & 8.33 \\
  1190002N15Rik & 165.63 & 9.02 \\
  Bcl2l14 & 165.71 & 7.54 \\
  Hist2h2be & 165.88 & 8.96 \\
  Gm6428 & 165.89 & 7.32 \\
  Dennd5b & 166.18 & 9.42 \\
  Gm12168 & 166.18 & 7.77 \\
  Gm12230 & 166.42 & 7.46 \\
  Acat3 & 166.61 & 7.17 \\
  Gpr20 & 166.84 & 8.41 \\
   \hline
  % \endgroup
  
  \label{tab:hot-annot}

\end{longtable}
}
%%%%% 3A end

%%%% 3B start
\section{Power analyses}

\subsection{Introduction}

The goal of this section is to characterize the statistical power of our pleiotropy test under a variety of conditions by studying a real data set. We examine pancreatic islet expression traits from the \citet{keller2018genetic} data. As in chapters 2 and 3A, we test only two traits at a time. Because we’ve chosen local expression traits in our analysis, we both know where each trait’s true QTL location (approximately), and we anticipate that each trait has a unique QTL that is distinct from QTL for other local expression traits. This design thus provides opportunities to study statistical power for our test.

We anticipate that inter-locus distance, univariate QTL strength, and correlation of founder allele effects patterns are three factors that contribute to power for our test. Specifically, we expect that greater inter-locus distance, greater univariate LOD scores, and less similar founder allele effects patterns correspond to greater statistical power to detect two separate QTL.

We use pancreatic islet gene expression traits from a publicly available data set, which \citet{keller2018genetic} first collected, analyzed, and shared. We examine a collection of 80 local traits on Chromosome 19 and perform our test for pleiotropy on pairs of traits. We also examine pairwise relationships among gene expression traits to characterize the impacts of univariate LOD score, inter-locus distance, and similarity of founder allele effects patterns on pleiotropy test statistics.



\subsection{Methods}


We analyzed data from 378 Diversity Outbred mice \citep{keller2018genetic}. \citet{keller2018genetic} genotyped tail biopsies with the GigaMUGA microarray \citep{morgan2016mouse}. They used RNA sequencing to measure genome-wide pancreatic islet cell gene expression for each mouse at the time of sacrifice \citep{keller2018genetic}. They shared these data, together with inferred founder allele probabilities, on the Data Dryad site (\url{https://datadryad.org/resource/doi:10.5061/dryad.pj105}). We performed analyses with the R statistical computing environment \citep{r} and the packages \texttt{qtl2} \citep{qtl2} and \texttt{qtl2pleio} \citep{qtl2pleio}.


We study below 80 Chromosome 19 local expression QTL and their corresponding transcript levels. We define a local expression QTL to be an expression QTL that is on the same chromosome as the gene itself. For example, the \emph{Asah2} gene is located on Chromosome 19 and its transcript levels have an expression QTL on Chromosome 19 (Table \ref{tab:ann4}). Thus, we term the Chromosome 19 \emph{Asah2} expression QTL a local expression QTL.

We choose to focus on local expression QTL, while ignoring nonlocal expression QTL, because we know, approximately, the true locations for local expression QTL. That is, a local expression QTL is near the corresponding gene position. Additionally, we expect that a given local expression QTL affects only one local expression trait. In our example above, we expect that the \emph{Asah2} expression QTL is near the \emph{Asah2} gene position and that no other local expression traits map to it.


Our design involves selection of a set of ``anchor'' expression traits. Gene \emph{Asah2} is located near the center of Chromosome 19 and has a very strong local expression QTL (Table \ref{tab:ann4}). We chose it as our first ``anchor'' gene expression trait. To diversify our collection of anchor genes, we chose three additional expression traits with local expression QTL. These three are \emph{Lipo1}, \emph{Lipo2}, and \emph{4933413C19Rik} (Table \ref{tab:ann4}). Together, the four anchor genes represent a variety of strong local expression trait LOD scores (from 60 to 101) and demonstrate modest variability in their founder allele effects (Table~\ref{tab:effects}). All four anchor genes are located near the middle of Chromosome 19 (Table~\ref{tab:ann4}).




We identified a set of 76 non-anchor local expression traits that map to the 20-Mb region centered on the peak for \emph{Asah2}, at 32.1 Mb. Each trait among the 76 maps to Chromosome 19 with a univariate LOD score of at least 10 (Table \ref{tab:ann76}).


% latex table generated in R 3.5.2 by xtable 1.8-3 package
% Thu Jan  3 16:29:15 2019
\begin{table}[ht]
\caption{Annotations for four anchor genes.}\label{tab:ann4}
\centering
\begin{tabular}{>{\em}lrrrr}
  \hline
Gene & Start & End & QTL peak position & LOD \\
  \hline
Asah2 & 31.98 & 32.06 & 32.14 & 101.20 \\
  Lipo1 & 33.52 & 33.76 & 33.67 & 85.46 \\
  Lipo2 & 33.72 & 33.76 & 33.02 & 77.21 \\
  4933413C19Rik & 28.58 & 28.58 & 28.78 & 60.41 \\
   \hline
\end{tabular}
\end{table}







For each Chromosome 19 marker, we estimated founder allele and covariate effects. We calculated:

\begin{equation}\label{eq:uni-Bhat}
\widehat {(B:C)} = \left((X:W)^T\hat\Gamma^{-1}(X:W)\right)^{-1}(X:W)^T\hat\Gamma^{-1}Y
\end{equation}

\noindent where $B:C$ denotes the concatenation of $B$ and $C$, and $X:W$ refers to the $n$ by $12$ matrix resulting from appending the columns of $W$ to the $X$ matrix. $\hat\Gamma$ is the covariance matrix defined by Equation \ref{eq:Sigma}.

\begin{equation}\label{eq:Sigma}
\hat \Gamma = \hat\sigma_g^2 K + \hat \sigma_e^2 I_n
\end{equation}

\noindent We denote the restricted maximum likelihood estimates of the variance components by $\hat \sigma_g^2$ and $\hat \sigma_e^2$.


For each of the 80 expression traits, we calculated fitted values for each subject with the estimated founder allele and covariate effects (Equation \ref{eq:uni-Bhat}). We then calculated correlations between fitted values for pairs of traits. Each pairing involved one anchor gene expression trait and one other gene expression trait.

We anticipated that more similar two traits' founder allele effects would correspond, on average, to smaller pleiotropy test statistics. We base this expectation on findings from \citet{macdonald2007joint} and \citet{king2012genetic}. \citet{macdonald2007joint} and \citet{king2012genetic} found that two traits that associate with a single pleiotropic QTL tended to have similar founder allele effects patterns for biallelic markers.

We performed two-dimensional QTL scans for $4 * 76 + \binom{4}{2} = 310$ pairs. Each pair included one of the four anchor gene expression traits and either one of 76 non-anchor gene expression traits or one of the remaining three anchor gene expression traits.

Our two-dimensional QTL scan encompassed a 1000 by 1000 marker grid from 18.1 Mb to 42.5 Mb on Chromosome 19. Each scan involved fitting 1000 x 1000 = 1,000,000 models via generalized least squares. For a given ordered pair of markers, we used the bivariate linear mixed effects model and methods defined in Chapter 2. These methods are implemented in the R package \texttt{qtl2pleio} \citep{qtl2pleio}.

For each of the 80 expression traits, we used the fitted founder allele and covariate effects ($\widehat{B:C}$ in Equation~\ref{eq:uni-Bhat}) to obtain fitted values vectors for every subject and all 80 traits (Equation~\ref{eq:fitted}). For each of the 310 pairings of traits, we calculated correlations among the fitted values vectors. Our motivation for working with the fitted values vectors (instead of the $\hat B$ estimated founder allele effects vectors) is that the fitted values approximately weight the allele effects by allele frequency. An alternative analysis might neglect the covariates when calculating fitted values.

\begin{equation}\label{eq:fitted}
\hat Y = X\hat B + W \hat C
\end{equation}


\subsection{Results}

% latex table generated in R 3.5.2 by xtable 1.8-3 package
% Sat Feb  2 10:17:32 2019
\begin{table}[ht]
\caption{Founder allele effect estimates at Chromosome 19 QTL peak position.}\label{tab:effects}
\centering
\begin{tabular}{>{\em}cccc}
  \hline
 Gene & Founder allele & Effect & Standard error \\
  \hline
\multirow{8}{*}{Asah2} & A & -0.96 & 0.17 \\
  & B & 1.01 & 0.19 \\
  & C & 0.14 & 0.17 \\
  & D & -1.16 & 0.17 \\
  & E & 1.05 & 0.16 \\
  & F & -0.61 & 0.20 \\
  & G & 1.81 & 0.16 \\
  & H & -0.18 & 0.18 \\
  \hline
  \multirow{8}{*}{Lipo1} & A & 0.29 & 0.18 \\
  & B & 0.13 & 0.21 \\
  & C & 0.28 & 0.20 \\
  & D & 0.23 & 0.19 \\
  & E & -0.17 & 0.18 \\
  & F & -0.28 & 0.21 \\
  & G & 2.55 & 0.19 \\
  & H & -0.72 & 0.19 \\
  \hline
\multirow{8}{*}{Lipo2} & A & -0.10 & 0.18 \\
  & B & -0.28 & 0.23 \\
  & C & 0.00 & 0.20 \\
  & D & 0.01 & 0.18 \\
  & E & -0.77 & 0.17 \\
  & F & -0.89 & 0.22 \\
  & G & 2.65 & 0.18 \\
  & H & -0.70 & 0.20 \\
   \hline
\multirow{8}{*}{4933413C19Rik} & A & 0.29 & 0.23 \\
  & B & 0.76 & 0.24 \\
  & C & 0.81 & 0.21 \\
  & D & 0.49 & 0.24 \\
  & E & 0.67 & 0.20 \\
  & F & -0.53 & 0.22 \\
  & G & -1.65 & 0.18 \\
  & H & 0.67 & 0.21 \\
   \hline
\end{tabular}
\end{table}


All four anchor traits demonstrate strong PWK (``G'') allele effects. \emph{Lipo2} and \emph{Asah2} have similar patterns among allele effects (at their respective QTL peaks) (Table~\ref{tab:effects}).


%\subsection{Pleiotropy likelihood ratio test statistic vs. chromosomal position}

\begin{figure}
    \centering
    \includegraphics[width = \textwidth]{../keller-2018-chr19-power/Rmd/lrt-v-middle-of-gene.pdf}
    \caption[Pleiotropy LRT vs. chromosomal position plots reveal that higher values of pleiotropy LRT tend to correspond to greater interlocus distance and greater univariate LOD score.]{Each anchor gene has its own panel. Along the horizontal axis is Chromosome 19 position. The vertical axis is for pleiotropy test statistic value. Each point corresponds to a local gene expression trait. Point color corresponds to the nonlocal gene's univariate LOD score, with lighter shades of blue denoting greater values of univariate LOD score. Vertical black bar denotes the anchor gene's position on Chromosome 19. All four panels reveal that points further from the anchor gene tend to show greater test statistic values. Additionally, the \emph{Lipo1} and \emph{Lipo2} panels offer an opportunity to compare the impact of anchor gene univariate LOD score on pleiotropy test statistic values.}
    \label{fig:middle}
\end{figure}

Each anchor gene has its own panel in Figure \ref{fig:middle}. Along the horizontal axis is Chromosome 19 position. The vertical axis is for pleiotropy test statistics. Each point corresponds to a local gene expression trait. Point color corresponds to the nonlocal gene's univariate LOD score, with lighter shades of blue denoting greater values of univariate LOD score. Vertical black bar denotes the anchor gene's position on Chromosome 19. All four panels reveal that points further from the anchor gene tend to show greater test statistic values. Additionally, because of their nearly identical positions, the \emph{Lipo1} and \emph{Lipo2} panels offer an opportunity to compare the impact of anchor gene univariate LOD score on pleiotropy test statistics.




%\subsection{Pleiotropy likelihood ratio test statistics vs. univariate LOD scores}

\begin{figure}
    \centering
    \includegraphics[width = \textwidth]{../keller-2018-chr19-power/Rmd/lrt-v-univariate-lod.pdf}
    \caption[Pleiotropy LRT vs. univariate LOD score plots reveal that greater univariate LOD scores (and greater interlocus distance) tend to correspond to greater pleiotropy LRT values.]{Vertical axis denotes pleiotropy test statistic value, while horizontal axis denotes univariate LOD score. Each point corresponds to a single gene expression trait. Panels correspond to the anchor gene expression trait. The pleiotropy test statistics correspond to analyses involving a single gene expression trait and the specified anchor gene expression trait.}
    \label{fig:lod}
\end{figure}

Analyses for all four anchor gene expression traits demonstrate that greater univariate LOD scores tend to correspond to greater values of the pleiotropy test statistic (Figure \ref{fig:lod}).








%\subsection{Pleiotropy likelihood ratio test statistics vs. fitted values correlation}

\begin{figure}
    \centering
    \includegraphics[width = \textwidth]{../keller-2018-chr19-power/Rmd/lrt-v-corr.pdf}
    \caption[Pleiotropy LRT vs. fitted values correlations plots reveal little evidence for a relationship.]{Vertical axis denotes the pleiotropy test statistic value, and horizontal axis indicates absolute value of the correlation between vectors of fitted values. Each point corresponds to a pairing between the specified anchor expression trait and one of the 79 other expression traits.}
    \label{fig:cor}
\end{figure}

Figure \ref{fig:cor} features four panels, one for each anchor gene. Each point corresponds to a pairing between the specified anchor and one of the 79 other gene expression traits.



\subsection{Discussion}

Our goal for this study was to characterize the impacts of univariate LOD score, inter-locus distance, and founder allele effects pattern similarities on pleiotropy test statistic values. Our study design, in which we examined 310 pairs of local gene expression traits on Chromosome 19, allowed us to interrogate both the effects of univariate association strength and the effects of inter-locus distance. We found that stronger univariate associations and greater inter-locus distances correspond to greater pleiotropy test statistic values (Figures \ref{fig:middle} and \ref{fig:lod}). We expected these trends based on our simulation studies in Chapter 2.

Figure \ref{fig:cor} revealed no marginal relationship between fitted values correlations and pleiotropy test statistics. However, close examination of Figure \ref{fig:cor} reveals the possibility that there is an interaction between 1) fitted values correlations and 2) univariate association strength. In every panel, those expression traits with stronger univariate associations tend to have steeper slopes between the conditional mean pleiotropy test statistic values and fitted values correlations. The plots suggest that, at greater univariate LOD values, there is a greater (negative) relationship between fitted values correlation and pleiotropy test statistic value.

We anticipated that more similar founder allele effects patterns would correspond to smaller values for the pleiotropy test statistic, when holding other factors constant. As we stated above, \citet{macdonald2007joint} and \citet{king2012genetic} argued that, for biallelic markers, two pleiotropic traits should have similar founder allele effects patterns. In our setting, it's unclear whether the markers are biallelic in the collection of eight founder lines.

We've demonstrated strong evidence in support of the roles of 1) univariate QTL LOD scores and 2) interlocus distances impacting pleiotropy test statistic values. Greater univariate QTL scores and greater interlocus distance lead to greater pleiotropy test statistics. Future research may clarify the impoact of founder allele effects patterns on pleiotropy test statistics. The fact that all four anchor traits had strong PWK effects limited our ability to fully define the impact of allele effects patterns on our test statistics.

Throughout this study, we elected to use test statistic values rather than p-values, as our measure of evidence supporting the separate QTL hypothesis. The primary reason for doing this is to avoid the computationally costly bootstrap sampling and two-dimensional QTL scans that we would need to get bootstrap p-values.

We share our analysis \texttt{R} code \citep{r} as a \texttt{git} repository at this URL: \url{https://github.com/fboehm/keller-2018-chr19-power}.

% latex table generated in R 3.5.2 by xtable 1.8-3 package
% Thu Jan  3 16:43:33 2019
\begin{table}[ht]
\caption{Annotations for 76 non-anchor genes on Chromosome 19.}\label{tab:ann76}
\centering
\begingroup\tiny
\begin{tabular}{>{\em}lrrrr}
  \hline
Gene & Start & End & Peak position & LOD \\
  \hline
C030046E11Rik & 29.52 & 29.61 & 29.55 & 95.58 \\
  Tctn3 & 40.60 & 40.61 & 40.59 & 90.00 \\
  Gm7237 & 33.41 & 33.42 & 33.67 & 74.61 \\
  Lipo4 & 33.50 & 33.52 & 34.00 & 68.23 \\
  Dock8 & 25.00 & 25.20 & 25.07 & 63.17 \\
  Sorbs1 & 40.30 & 40.40 & 40.48 & 61.89 \\
  Lipm & 34.10 & 34.12 & 34.06 & 58.43 \\
  Blnk & 40.93 & 40.99 & 40.76 & 57.16 \\
  A830019P07Rik & 35.84 & 35.92 & 35.60 & 55.54 \\
  Uhrf2 & 30.03 & 30.09 & 29.96 & 54.40 \\
  Mbl2 & 30.23 & 30.24 & 30.18 & 52.81 \\
  Myof & 37.90 & 38.04 & 38.05 & 48.46 \\
  Gm27042 & 40.59 & 40.59 & 40.61 & 44.27 \\
  Btaf1 & 36.93 & 37.01 & 36.90 & 41.25 \\
  Hoga1 & 42.05 & 42.07 & 42.09 & 41.23 \\
  Ppp1r3c & 36.73 & 36.74 & 36.53 & 40.69 \\
  Pcgf5 & 36.38 & 36.46 & 36.24 & 40.06 \\
  Slc35g1 & 38.40 & 38.41 & 38.35 & 38.11 \\
  Pten & 32.76 & 32.83 & 32.77 & 37.95 \\
  Gldc & 30.10 & 30.18 & 30.17 & 36.26 \\
  Lgi1 & 38.26 & 38.31 & 38.17 & 34.91 \\
  C330002G04Rik & 23.04 & 23.08 & 23.34 & 34.84 \\
  Ppapdc2 & 28.96 & 28.97 & 29.09 & 34.71 \\
  Gm8978 & 33.61 & 33.63 & 33.03 & 34.59 \\
  Mms19 & 41.94 & 41.98 & 41.98 & 32.03 \\
  Ankrd22 & 34.12 & 34.17 & 34.04 & 31.83 \\
  Cdc37l1 & 28.99 & 29.02 & 29.03 & 31.14 \\
  Sgms1 & 32.12 & 32.39 & 32.11 & 30.10 \\
  Entpd1 & 40.61 & 40.74 & 40.50 & 29.73 \\
  Cbwd1 & 24.92 & 24.96 & 24.73 & 29.65 \\
  Gm14446 & 34.59 & 34.60 & 34.28 & 27.65 \\
  Ermp1 & 29.61 & 29.65 & 29.70 & 26.57 \\
  Gm9938 & 23.72 & 23.73 & 23.87 & 26.46 \\
  Insl6 & 29.32 & 29.33 & 29.37 & 26.23 \\
  Slc16a12 & 34.67 & 34.75 & 34.71 & 25.54 \\
  Pgm5 & 24.68 & 24.86 & 25.00 & 24.30 \\
  Morn4 & 42.07 & 42.09 & 41.79 & 23.86 \\
  Exosc1 & 41.92 & 41.93 & 42.10 & 23.28 \\
  Smarca2 & 26.61 & 26.78 & 26.59 & 23.25 \\
  4930418C01Rik & 24.42 & 24.43 & 23.92 & 23.10 \\
  2700046G09Rik & 32.39 & 32.39 & 32.25 & 23.02 \\
  Kcnv2 & 27.32 & 27.34 & 27.14 & 22.88 \\
  1500017E21Rik & 36.61 & 36.71 & 37.07 & 22.78 \\
  Fra10ac1 & 38.19 & 38.22 & 38.35 & 22.48 \\
  Rnls & 33.14 & 33.39 & 34.17 & 21.94 \\
  Noc3l & 38.79 & 38.82 & 40.20 & 21.67 \\
  Pip5k1b & 24.29 & 24.56 & 24.15 & 21.62 \\
  Plgrkt & 29.35 & 29.37 & 29.37 & 20.65 \\
  Ifit3 & 34.58 & 34.59 & 34.28 & 20.45 \\
  Fas & 34.29 & 34.33 & 34.20 & 19.65 \\
  Slit1 & 41.60 & 41.74 & 41.70 & 18.95 \\
  Rrp12 & 41.86 & 41.90 & 41.71 & 18.09 \\
  Ak3 & 29.02 & 29.05 & 29.55 & 16.90 \\
  A1cf & 31.87 & 31.95 & 32.11 & 15.56 \\
  4430402I18Rik & 28.90 & 28.97 & 29.37 & 15.43 \\
  Pdlim1 & 40.22 & 40.27 & 40.25 & 15.25 \\
  Gm26902 & 34.47 & 34.48 & 36.15 & 14.26 \\
  Plce1 & 38.48 & 38.79 & 38.42 & 14.26 \\
  Slc1a1 & 28.84 & 28.91 & 28.97 & 14.18 \\
  Fam122a & 24.48 & 24.48 & 24.08 & 14.07 \\
  Lipa & 34.49 & 34.53 & 34.29 & 14.06 \\
  Mamdc2 & 23.30 & 23.45 & 23.35 & 13.12 \\
  Kif11 & 37.38 & 37.42 & 37.33 & 12.93 \\
  4933411K16Rik & 42.05 & 42.05 & 42.08 & 12.92 \\
  Ccnj & 40.83 & 40.85 & 40.59 & 12.19 \\
  Gm340 & 41.58 & 41.59 & 41.30 & 12.17 \\
  Fxn & 24.26 & 24.28 & 24.31 & 12.07 \\
  Stambpl1 & 34.19 & 34.24 & 34.28 & 11.62 \\
  Pde6c & 38.13 & 38.18 & 38.07 & 11.54 \\
  Cyp26a1 & 37.70 & 37.70 & 37.48 & 11.35 \\
  Ch25h & 34.47 & 34.48 & 32.50 & 10.74 \\
  Pank1 & 34.81 & 34.88 & 35.55 & 10.61 \\
  9930021J03Rik & 29.71 & 29.81 & 28.71 & 10.32 \\
  Klf9 & 23.14 & 23.17 & 23.34 & 10.26 \\
  Ubtd1 & 41.98 & 42.03 & 41.71 & 10.25 \\
  Lipk & 34.01 & 34.05 & 34.29 & 10.23 \\
   \hline
\end{tabular}
\endgroup
\end{table}


%%%% 3B end

%%%% 3C start

\section{Microbiome case study}
\subsection{Introduction}

Recent technological innovations have fueled exploration of ecological relationships between gut microbiota and their hosts. Advances in mass spectrometry experimental methods have enabled high-throughput quantification of lipid levels and protein concentrations. These developments, when coupled with experiments to quantify the gut microbiome, have the potential to uncover new microbiome-host interactions. Such discoveries would lead to a more nuanced understanding of organismal biology and health implications of the gut microbiome. Below, we use our test of pleiotropy vs. separate QTL to identify a pleiotropic QTL that affects both the abundance level of a group of bacteria in the distal gut and plasma cholic acid levels in the host. While many questions remain after our investigation, our identification of a single pleiotropic locus that affects these two phenotypes is an important preliminary step for further investigations.  \todo[inline]{add citations to reviews here. Look at J's ms too. What does J discuss in her intro?}



\subsection{Methods}

As we describe in Kreznar et al. (forthcoming), we analyzed data from 384 Diversity Outbred mice. \citet{keller2018genetic} analyzed data from many of these same mice. Specifically, we examined two microbiome-related phenotypes, 1. plasma cholic acid and 2. \emph{Turicibacter} abundance in the distal gut. Both traits map to Chromosome 8 (4.3 Mb and 5.7 Mb, respectively) in univariate QTL scans (Table \ref{tab:3c-lod-peaks}).

Following the procedures outlined by Kreznar et al. (forthcoming), we collected fecal samples from all mice immediately immediately after a four-hour fast. We sacrificed the mice, at the age of 22 weeks, immediately after the fast. We extracted DNA from fecal samples and subjected it to 16S ribosomal RNA gene sequencing, as we describe in Kreznar et al. (forthcoming) in efforts to infer abundances of microbial taxa in the fecal samples. Demultiplexed, paired-end FASTQ files resulted from the sequencing. We used the QIIME2 (version 2018.4) software package for quality control and processing of sequence data. We applied the DADA2 software package to denoise sequencing reads and to identify de novo sub-operational taxonomic units. We aligned sequence variants with the software package mafft. After FastTree-based phylogeny reconstruction, we assigned taxonomic classifications with classify-sklearn against the Greengenes OTUs reference sequences. We normalized sequencing data with cumulative sum scaling with MetagenomeSeq. We limited study of microbiome-derived traits to those that we detected in at least 20\% of subjects. The \emph{Turicibacter} abundance trait is one element of the resulting core measurable microbiota. 

As Kreznar et al. (forthcoming) describe, we collected 40 $\mu$L of plasma from each mouse at time of sacrifice. After removing soluble proteins, we analyzed each mouse's sample by mass spectrometry. Specifically, we performed liquid chromatography - mass spectrometry under conditions specified in Kreznar et al. to measure abundances of pre-specified bile acids. We processed and normalized plasma bile acid measurements to obtain the phenotypes for QTL analysis.

We inferred 36-state genotype probabilities for every (autosomal) marker and every mouse from GigaMUGA SNP microarray data \citep{morgan2015mouse}. We used a hidden Markov model strategy developed by \citet{broman2012genotype, broman2012haplotype} and implemented in the R package \texttt{qtl2} \citep{qtl2}. We treated these inferred genotype probabilities as known quantities in the analyses below.

We calculated founder allele dosages by summing the appropriate genotype probabilities. We performed these calculations with functions from the R package \texttt{qtl2}\citep{qtl2}.

We performed univariate QTL analyses for both microbiome-related traits. We used the R package \texttt{qtl2} for these analyses \citep{qtl2}.

Specifically, for each marker and each univariate phenotype $Y$, we fitted the model:

\begin{equation}
Y = XB + WC + G + E  
\label{eq:uni-model}
\end{equation}

where $Y$ is a n by 1 vector with trait values, $X$ is a $n$ by $8$ matrix of founder allele probabilities for a single genetic marker, $B$ is a $8$ by $1$ vector of unknown founder allele effects, $W$ is a $n$ by $c$ matrix of covariates, $C$ is a $c$ by $1$ vector of covariate effects, $G$ is a $n$ by $1$ vector of random effects, and $E$ is a $n$ by $1$ vector of random errors. $G$ and $E$ are independent, and 

\begin{equation}
G \sim N(0, \sigma^2_g K)    
\end{equation}
\begin{equation}
E \sim N(0, \sigma^2_e I)    
\end{equation}


In the above expressions, $K$ is a kinship matrix, while $\sigma^2_g$ and $\sigma^2_e$ are unknown constants. We used a distinct kinship matrix for each chromosome via the leave-one-chromosome-out method \citep{yang2014advantages}. By using leave-one-chromosome-out kinship matrices, we ensured that each marker's genetic data was not used in both the $X$ matrix and the random effects in a single model. \citet{yang2014advantages} demonstrated a decrease in statistical power to detect genetic associations when testing a marker that is used in constructing the kinship matrix. 

We then calculated LOD values comparing the univariate models' base-10 log-likelihoods  at each marker to the base-10 log-likelihood of the model without founder allele dosages, \emph{i.e.}, the model with a single column of $1$'s replacing the eight columns of founder allele dosages. In summarizing LOD peak results, we identified a small region on Chromosome 8 that contains LOD peaks for both \emph{Turicibacter} abundance and plasma cholic acid levels.


With the founder allele dosages, we calculated 20 kinship matrices, one for each chromosome, via the leave-one-chromosome-out method \citep{yang2014advantages}. Specifically, we calculated the $(i, j)$ entry in the matrix $K$ with Equation \ref{eq:kinship}.

\begin{equation}
K[i, j] = \sum_{k,l}p_{ikl}p_{jkl}
\label{eq:kinship}
\end{equation}
where $p_{ikl}$ and $p_{jkl}$ are the dosages for founder allele $l$ at marker $k$ for subjects $i$ and $j$ and the sum is over all markers that are not present on a given chromosome. Thus, we omitted Chromosome 8 markers when calculating the kinship matrix for use in our test below. 



We estimated founder allele effects using the above model. We calculated

\begin{equation}
\widehat{(B:C)} = \left((X:W)^T\Gamma^{-1}(X:W)\right)^{-1}(X:W)^T\Gamma^{-1}Y
\label{eq:allele-effects}
\end{equation}

where $B:C$ denotes the concatenation of $B$ vector and $C$ vector and $X:W$ denotes the appending of the eight columns of $X$ with the $c$ columns of $W$ into a matrix with $(8 + c)$ columns. 

We define covariance matrix $\Gamma$ in Equation \ref{eq:gamma}.

\begin{equation}
\Gamma = \sigma^2_g K + \sigma^2_e I_n
\label{eq:gamma}
\end{equation}


We performed a test of pleiotropy vs. separate QTL for our two traits, \emph{Turicibacter} abundance and plasma cholic acid levels. We proceeded by estimating variance components, performing the two-dimensional scan, calculating our likelihood ratio test statistic, constructing bootstrap samples, performing two-dimensional scans for each bootstrap sample, and then determining a bootstrap p-value for our test statistic.



%\subsubsection{Estimating variance components}

Estimation of variance components $V_g$ and $V_e$, both of which are 2 by 2 covariance matrices, proceeded by an expectation-maximization algorithm as in the GEMMA algorithm \citep{zhou2014efficient}. We used our implementation of the GEMMA software's algorithm for multivariate linear mixed effects models in the R package \texttt{gemma2} \citep{gemma2}. Unlike the GEMMA algorithm, which relies on both expectation-maximization and Newton-Raphson algorithms, \texttt{gemma2} uses only expectation-maximization to estimate the variance components. \citet{zhou2014efficient} explain that the Newton-Raphson algorithm decreased the computation time compared to using expectation-maximization alone. Because we perform only one instance of variance component estimation for each analysis, we decided to implement only the expectation-maximization algorithm. 

We omitted founder allele probabilities when estimating variance components. That is, we used the model:

\begin{equation}
vec(Y) = X_1 vec(B) + vec(G) + vec(E)
\label{eq:no-geno}
\end{equation}

with 

\begin{equation}
G \sim MN(0, K, V_g)
\end{equation}

\begin{equation}
E \sim MN(0, I_n, V_e)
\end{equation}

We assume that $G$ and $E$ are independent. $X_1$ is a $2n$ by $2(c + 1)$ matrix containing two $n$ by $(c + 1)$ nonzero blocks on the diagonal. The two off-diagonal blocks have all entries set to zero. Each $n$ by $(c + 1)$ block contains $c$ covariates and a column of $1$'s. $vec$ denotes the operation of stacking columns of a matrix into a matrix with exactly one column. Since $Y$ is a $n$ by $2$ matrix, $vec(Y)$ is a $2n$ by $1$ column matrix.


With the estimated variance components, which we obtained from model \ref{eq:no-geno}, we calculated the the covariance matrix $\Sigma$.

\begin{equation}
\Sigma = V_g \otimes K + V_e \otimes I_n
\end{equation}

We use the matrix $\Sigma$ in our two-dimensional QTL scan below.



%\subsubsection{Two-dimensional QTL scan}

For every ordered pair of markers in the scan region, we fitted by generalized least squares model \ref{eq:biv-model}.

\begin{equation}
vec(Y) = X vec(B) + vec(G) + vec(E)
\label{eq:biv-model}
\end{equation}

with 

\begin{equation}
G \sim MN(0, K, V_g)
\end{equation}

\begin{equation}
E \sim MN(0, I_n, V_e)
\end{equation}

and $G$ and $E$ independent. $X$ is a $2n$ by $2(8 + c)$ block-diagonal matrix containing two $n$ by $(8 + c)$ nonzero blocks. Each $n$ by $(8 + c)$ block on the diagonal contains $8$ founder allele dosages and $c$ covariate values for each subject.  

Our R package \texttt{qtl2pleio} contains the function \texttt{scan\_pvl}, which performs a two-dimensional QTL scan. 

For each ordered pair of markers in our genomic region, we calculate a log-likelihood based on the multivariate normal distribution (Equation \ref{eq:ll}).

\begin{equation}
ll(\hat B | V_g, V_e, X, Y) = det\left(2\pi\Sigma\right)^{-\frac{1}{2}}\exp{\left(- \frac{1}{2}(vec(Y) - Xvec(\hat B))^T\Sigma^{-1}(vec(Y) - Xvec(\hat B))\right)}
\label{eq:ll}
\end{equation}



%\subsubsection{Calculating the test statistic}

The output of the two-dimensional QTL scan is a collection of log-likelihood values for each ordered pair of markers in the genomic region. To calculate our test statistic, we identify the maximum log-likelihood over the entire collection of values and the maximum over the restricted collection of log-likelihood values. The restricted set of log-likelihood values is those for which the first and second markers are the same. This corresponds to the null hypothesis of a single pleiotropic QTL for our test.

\begin{equation}
\Lambda = \log\left(\frac{\max_{\text{pleiotropy}}ll(\hat B)}{max_{\text{separate QTL}}ll(\hat B)}\right)
\end{equation}


%\subsubsection{Constructing parametric bootstrap samples}

To construct bootstrap samples, we first used the pleiotropy trace from the LOD profile plot to identify the marker with the greatest log-likelihood value. This was the eighth marker on Chromosome 8. We then performed a parametric bootstrap analysis to acquire 1000 bootstrap samples. Specifically, we used the parameter estimates for our model at the 8th marker on Chromosome 8 and took pseudo-random draws from the multivariate normal distribution with mean $Xvec(\hat B)$ and covariance matrix $\Sigma$. Our $2n$ by $2(8 + c)$ matrix $X$ contains two nonzero $n$ by $(8 + c)$ blocks on the diagonal. These two blocks are identical and both contain the founder allele dosages at the eighth marker on Chromosome 8, along with the $c$ covariates. 

Equivalently, we define this $X$ matrix as the Kronecker product of two smaller matrices (Equation \ref{eq:X8}). 

\begin{equation}
X = I_2 \otimes X_{\text{marker 8}}
\label{eq:X8}
\end{equation}

where $X_{\text{marker 8}}$ is the $n$ by $(8 + c)$ matrix of eight founder allele dosages and $c$ covariates at the eighth marker on Chromosome 8, and $I_2$ is the 2 by 2 identity matrix.

Each bootstrap sample consists of a $n$ by $2$ matrix of two simulated phenotypes. For each of the 1000 bootstrap samples, we estimated variance components and performed a two-dimensional QTL scan. We then calculated the pleiotropy vs. separate QTL likelihood ratio test statistic for each bootstrap sample.


%\subsubsection{Determining the bootstrap p-value}

With the collection of 1000 test statistic values from the bootstrap samples, we calculated a bootstrap p-value as the proportion of the 1000 tests, denoted by $\Lambda_i$, with $i$ ranging from 1 to 1000, that were at least as extreme as the true test statistic $\Lambda$ (Equation \ref{eq:bootp}).

\begin{equation}
\text{bootstrap p-value} = \frac{\#(\Lambda_i \geq \Lambda)}{1000}
\label{eq:bootp}
\end{equation}



\subsection{Results}

%\subsection{Scatter plot for two phenotypes}

\begin{figure}
\includegraphics{../kemis-do-analysis/Rmd/scatter.pdf}
\caption{Scatter plot of Plasma cholic acid levels against \emph{Turicibacter} abundance}
\end{figure}

The scatter plot of 

%\subsection{Univariate QTL analyses}

% latex table generated in R 3.5.1 by xtable 1.8-3 package
% Fri Nov 30 10:14:43 2018
\begin{table}[ht]
\centering
\begin{tabular}{lrrr}
  \hline
phenotype & chromosome & position & LOD \\ 
  \hline
plasma cholic acid &   1 & 91.61 & 5.37 \\ 
 \emph{Turicibacter} abundance &   2 & 17.20 & 5.40 \\ 
  plasma cholic acid &   3 & 40.53 & 5.80 \\ 
  plasma cholic acid &   7 & 122.19 & 6.83 \\ 
  plasma cholic acid &   8 & 4.32 & 6.52 \\ 
 \emph{Turicibacter} abundance &   8 & 5.68 & 7.22 \\ 
  plasma cholic acid &  12 & 16.60 & 5.24 \\ 
 \emph{Turicibacter} abundance &  12 & 76.34 & 5.17 \\ 
   \hline
\end{tabular}
\caption{\label{tab:3c-lod-peaks}Genome-wide LOD peaks greater than 5 for plasma cholic acid levels and \emph{Turicibacter} abundance. Both traits map to approximately the same region on Chromosome 8.}
\end{table}

\begin{figure}

\begin{subfigure}[b]{\textwidth}
\centering
\includegraphics[height = 3.5in]{../kemis-do-analysis/Rmd/Chol_effects.pdf}
\subcaption{}
\end{subfigure}

\begin{subfigure}[b]{\textwidth}
\centering
\includegraphics[height = 3.5in]{../kemis-do-analysis/Rmd/Turc_effects.pdf}
\subcaption{}
\end{subfigure}
\caption{}
\end{figure}



%\subsection{Profile LOD plot}

\begin{figure}
\includegraphics{../kemis-do-analysis/Rmd/profiles.pdf}
\caption{Profile LODs for \emph{Turicibacter} abundance and Plasma cholic acid levels.}
\end{figure}


%\subsection{Calculating the likelihood ratio test statistic and bootstrap-based p-value}

We used the R package \texttt{qtl2pleio} \citep{qtl2pleio} to calculate the likelihood ratio test statistic, $\Lambda = 0.45$. We determined the bootstrap p-value to be $0.531$. Thus, we failed to reject the null hypothesis of pleiotropy.


\subsection{Discussion}

The above study illustrates another scientific application of our test of pleiotropy vs. separate QTL. We plan to perform additional statistical analyses, including mediation studies, with these data. Our test of pleiotropy vs. separate QTL tells us that the data are consistent with presence of a single QTL affecting both \emph{Turicibacter} abundance and plasma cholic acid levels, but it doesn't provide information about the possibilities that plasma cholic acid levels affect \emph{Turicibacter} abundance or that \emph{Turicibacter} abundance affects plasma cholic acid levels.

\todo{incorporate what is known about cholic acid biology here. Is cholic acid absorbed from the distal gut? What is needed for cholic acid synthesis? What is the chemical structure of cholic acid? Look at Julia's ms}

%%%% end 3C

%%% Ch4 start
\chapter{Computing vignettes}


%%%%% recla-analysis vignette
\section{Recla data analysis vignette}
\begin{Shaded}
\begin{Highlighting}[]
\KeywordTok{library}\NormalTok{(dplyr)}
\KeywordTok{library}\NormalTok{(ggplot2)}
\KeywordTok{library}\NormalTok{(qtl2pleio)}
\KeywordTok{library}\NormalTok{(qtl2)}
\NormalTok{knitr}\OperatorTok{::}\NormalTok{opts_chunk}\OperatorTok{$}\KeywordTok{set}\NormalTok{(}\DataTypeTok{tidy =} \OtherTok{TRUE}\NormalTok{, }
                      \DataTypeTok{tidy.opts =} \KeywordTok{list}\NormalTok{(}\DataTypeTok{width.cutoff =} \DecValTok{60}\NormalTok{)}
\NormalTok{                      )}
\end{Highlighting}
\end{Shaded}

\hypertarget{load-recla-from-qtl2data-github-repository}{%
\subsection{\texorpdfstring{Load Recla from \texttt{qtl2data} github
repository}{Load Recla from qtl2data github repository}}\label{load-recla-from-qtl2data-github-repository}}

\begin{Shaded}
\begin{Highlighting}[]
\NormalTok{file <-}\StringTok{ }\KeywordTok{paste0}\NormalTok{(}\StringTok{"https://raw.githubusercontent.com/rqtl/"}\NormalTok{,}
               \StringTok{"qtl2data/master/DO_Recla/recla.zip"}\NormalTok{)}
\NormalTok{recla <-}\StringTok{ }\KeywordTok{read_cross2}\NormalTok{(file)}
\CommentTok{# make sex a covariate for use in qtl2pleio::scan_pvl}
\NormalTok{recla[[}\DecValTok{6}\NormalTok{]][ , }\DecValTok{1}\NormalTok{, drop =}\StringTok{ }\OtherTok{FALSE}\NormalTok{] ->}\StringTok{ }\NormalTok{sex}
\CommentTok{# insert pseudomarkers}
\KeywordTok{insert_pseudomarkers}\NormalTok{(recla, }\DataTypeTok{step =} \FloatTok{0.10}\NormalTok{) ->}\StringTok{ }\NormalTok{pseudomap}
\NormalTok{gm <-}\StringTok{ }\NormalTok{pseudomap}\OperatorTok{$}\StringTok{`}\DataTypeTok{8}\StringTok{`}
\end{Highlighting}
\end{Shaded}

\begin{Shaded}
\begin{Highlighting}[]
\NormalTok{probs <-}\StringTok{ }\KeywordTok{calc_genoprob}\NormalTok{(recla, }\DataTypeTok{map =}\NormalTok{ pseudomap)}
\end{Highlighting}
\end{Shaded}

We now convert the genotype probabilities to haplotype dosages.

\begin{Shaded}
\begin{Highlighting}[]
\NormalTok{aprobs <-}\StringTok{ }\KeywordTok{genoprob_to_alleleprob}\NormalTok{(probs)}
\end{Highlighting}
\end{Shaded}

We now calculate kinship matrices, by the ``leave one chromosome out
(loco)'' method.

\begin{Shaded}
\begin{Highlighting}[]
\NormalTok{kinship <-}\StringTok{ }\KeywordTok{calc_kinship}\NormalTok{(aprobs, }\StringTok{"loco"}\NormalTok{)}
\end{Highlighting}
\end{Shaded}

\begin{Shaded}
\begin{Highlighting}[]
\NormalTok{recla}\OperatorTok{$}\NormalTok{pheno ->}\StringTok{ }\NormalTok{ph}
\KeywordTok{log}\NormalTok{(ph) ->}\StringTok{ }\NormalTok{lph}
\KeywordTok{apply}\NormalTok{(}\DataTypeTok{FUN =}\NormalTok{ broman}\OperatorTok{::}\NormalTok{winsorize, }\DataTypeTok{X =}\NormalTok{ lph, }\DataTypeTok{MARGIN =} \DecValTok{2}\NormalTok{) ->}\StringTok{ }\NormalTok{wlph}
\CommentTok{#colnames(wlph)[c(7, 10, 22)] <- c("distance traveled in light", "percent time in light", "hot plate latency")}

\KeywordTok{as_tibble}\NormalTok{(wlph) ->}\StringTok{ }\NormalTok{wlph_tib}
\end{Highlighting}
\end{Shaded}

We next perform the univariate QTL scan for our phenotypes.

\begin{Shaded}
\begin{Highlighting}[]
\NormalTok{sex2 <-}\StringTok{ }\KeywordTok{matrix}\NormalTok{(}\KeywordTok{as.numeric}\NormalTok{(sex }\OperatorTok{==}\StringTok{ "female"}\NormalTok{), }\DataTypeTok{ncol =} \DecValTok{1}\NormalTok{)}
\KeywordTok{colnames}\NormalTok{(sex2) <-}\StringTok{ "female"}
\KeywordTok{rownames}\NormalTok{(sex2) <-}\StringTok{ }\KeywordTok{rownames}\NormalTok{(aprobs[[}\DecValTok{1}\NormalTok{]])}
\NormalTok{out <-}\StringTok{ }\KeywordTok{scan1}\NormalTok{(}\DataTypeTok{genoprobs =}\NormalTok{ aprobs, }\DataTypeTok{pheno =}\NormalTok{ wlph, }\DataTypeTok{kinship =}\NormalTok{ kinship, }\DataTypeTok{addcovar =}\NormalTok{ sex2, }\DataTypeTok{reml =} \OtherTok{TRUE}\NormalTok{)}
\end{Highlighting}
\end{Shaded}

Let's find the univariate QTL peaks for all phenotypes.

We want to look closely at those peaks on Chromosome 8. We'll save the
positions of peaks for our two traits of interest.

\begin{Shaded}
\begin{Highlighting}[]
\NormalTok{(peaks <-}\StringTok{ }\KeywordTok{find_peaks}\NormalTok{(out, pseudomap, }\DataTypeTok{threshold =} \DecValTok{5}\NormalTok{) }\OperatorTok
\StringTok{  }\KeywordTok{arrange}\NormalTok{(chr, pos) }\OperatorTok
\StringTok{   }\KeywordTok{select}\NormalTok{(}\OperatorTok{-}\StringTok{ }\NormalTok{lodindex))}
\end{Highlighting}
\end{Shaded}

\begin{verbatim}
##                    lodcolumn chr     pos      lod
## 1                         bw   1 23.9075 5.471580
## 2         OF_distance_first4   1 43.2385 5.772977
## 3             LD_transitions   1 95.8075 5.028258
## 4                OF_distance   2 49.9770 5.544458
## 5                         bw   2 52.3932 7.352334
## 6            OF_immobile_pct   2 53.2646 9.771813
## 7  VC_bottom_distance_first4   2 71.0160 6.531654
## 8         OF_distance_first4   3 10.7360 5.541166
## 9  VC_bottom_distance_first4   3 16.3700 5.518637
## 10           VC_top_time_pct   3 17.9390 5.951067
## 11         LD_distance_light   3 23.4390 5.203246
## 12                        bw   3 24.8390 5.632714
## 13        VC_top_time_first4   3 48.1280 6.144073
## 14           VC_top_velocity   3 48.5630 6.264824
## 15        VC_top_time_first4   4  3.5340 5.016970
## 16             OF_corner_pct   4  9.0111 6.272996
## 17           OF_immobile_pct   4 37.5206 5.529985
## 18         LD_distance_light   4 71.2992 5.040185
## 19                        bw   5 10.0740 5.728865
## 20 VC_bottom_distance_first4   5 19.9741 5.419061
## 21        VC_top_time_first4   5 20.0741 6.075282
## 22        VC_bottom_distance   5 20.5930 5.691913
## 23        VC_bottom_time_pct   5 20.5930 6.360001
## 24       TS_latency_immobile   5 43.3504 5.995115
## 25             OF_corner_pct   5 64.2551 5.665658
## 26           OF_immobile_pct   6 53.4292 6.968771
## 27     TS_frequency_climbing   6 57.0362 5.361091
## 28                        bw   7  9.1778 6.057098
## 29          TS_time_immobile   7 49.4778 8.067612
## 30        VC_bottom_distance   7 54.5591 5.011113
## 31        OF_distance_first4   7 57.9454 5.327302
## 32           VC_top_distance   7 83.8778 5.724788
## 33           OF_immobile_pct   7 83.9778 5.823556
## 34     TS_frequency_climbing   8 48.1732 5.483064
## 35         LD_distance_light   8 55.2762 5.323391
## 36              LD_light_pct   8 55.2762 5.274185
## 37                HP_latency   8 57.7732 6.223739
## 38         LD_distance_light   9 36.6965 5.196968
## 39              LD_light_pct   9 36.6965 5.417419
## 40        VC_top_time_first4   9 38.4834 5.109813
## 41           VC_top_time_pct   9 39.2680 6.356432
## 42                HP_latency   9 46.8502 5.222074
## 43                        bw  10  3.7781 6.526199
## 44        OF_distance_first4  10 29.6698 5.462975
## 45        VC_bottom_time_pct  10 32.5438 5.432804
## 46          OF_periphery_pct  10 74.8530 5.246487
## 47           VC_top_distance  11  7.8200 6.245803
## 48    VC_top_distance_first4  11 11.6236 5.486915
## 49 VC_bottom_distance_first4  11 54.3420 5.367052
## 50            LD_transitions  11 58.9000 5.903217
## 51     VC_bottom_transitions  11 60.5984 5.114051
## 52              LD_light_pct  11 63.3943 6.464176
## 53         LD_distance_light  11 63.4514 6.373437
## 54           VC_top_time_pct  12 20.5776 6.950144
## 55        VC_bottom_velocity  12 21.7760 5.653292
## 56             OF_center_pct  12 35.5140 6.399422
## 57                HP_latency  12 43.5150 5.131074
## 58          OF_periphery_pct  12 53.5776 7.240951
## 59             OF_corner_pct  13 59.7966 6.594594
## 60     VC_bottom_time_first4  14 11.9183 5.204369
## 61 VC_bottom_distance_first4  14 12.5316 5.763233
## 62     VC_bottom_transitions  14 12.5316 6.478533
## 63           VC_top_velocity  14 12.7819 6.840412
## 64        VC_bottom_distance  14 14.5316 5.592030
## 65     TS_frequency_climbing  14 21.1141 5.372594
## 66             OF_center_pct  14 53.7316 5.377182
## 67     TS_frequency_climbing  15 12.6680 6.043265
## 68              LD_light_pct  15 15.2374 5.674922
## 69        OF_distance_first4  16 23.2656 5.242101
## 70           VC_top_distance  17 15.6390 6.669275
## 71           VC_top_velocity  18  8.3750 5.558628
## 72        VC_top_time_first4  18 17.8068 6.246206
## 73            LD_transitions  18 37.4182 5.090332
## 74 VC_bottom_distance_first4  19 24.9615 7.362557
## 75     VC_bottom_time_first4  19 24.9615 7.499959
## 76           OF_immobile_pct  19 31.9505 5.639812
## 77                HP_latency  19 47.7977 5.485000
\end{verbatim}

\begin{Shaded}
\begin{Highlighting}[]
\NormalTok{peaks8 <-}\StringTok{ }\NormalTok{peaks }\OperatorTok
\StringTok{  }\KeywordTok{filter}\NormalTok{(chr }\OperatorTok{==}\StringTok{ }\DecValTok{8}\NormalTok{, pos }\OperatorTok{>}\StringTok{ }\DecValTok{50}\NormalTok{, pos }\OperatorTok{<}\StringTok{ }\DecValTok{60}\NormalTok{)}
\NormalTok{pos_LD_light_pct <-}\StringTok{ }\NormalTok{peaks8 }\OperatorTok
\StringTok{  }\KeywordTok{filter}\NormalTok{(lodcolumn }\OperatorTok{==}\StringTok{ "LD_light_pct"}\NormalTok{) }\OperatorTok
\StringTok{  }\KeywordTok{select}\NormalTok{(pos)}
\NormalTok{pos_HP_latency <-}\StringTok{ }\NormalTok{peaks8 }\OperatorTok
\StringTok{  }\KeywordTok{filter}\NormalTok{(lodcolumn }\OperatorTok{==}\StringTok{ "HP_latency"}\NormalTok{) }\OperatorTok
\StringTok{  }\KeywordTok{select}\NormalTok{(pos)}
\end{Highlighting}
\end{Shaded}

\hypertarget{correlation}{%
\subsection{Correlation}\label{correlation}}

Given that the two traits ``percent time in light'' and ``distance
traveled in light'' share a peak, we want to ask how correlated they
are.

\begin{Shaded}
\begin{Highlighting}[]
\KeywordTok{cor}\NormalTok{(wlph[ , }\DecValTok{7}\NormalTok{], wlph[ , }\DecValTok{10}\NormalTok{], }\DataTypeTok{use =} \StringTok{"complete.obs"}\NormalTok{)}
\end{Highlighting}
\end{Shaded}

\begin{verbatim}
## [1] 0.8859402
\end{verbatim}

\begin{Shaded}
\begin{Highlighting}[]
\KeywordTok{cor}\NormalTok{(wlph[ , }\DecValTok{22}\NormalTok{], wlph[ , }\DecValTok{10}\NormalTok{], }\DataTypeTok{use =} \StringTok{"complete.obs"}\NormalTok{)}
\end{Highlighting}
\end{Shaded}

\begin{verbatim}
## [1] -0.1507317
\end{verbatim}

\begin{Shaded}
\begin{Highlighting}[]
\KeywordTok{cor}\NormalTok{(wlph[ , }\DecValTok{7}\NormalTok{], wlph[ , }\DecValTok{22}\NormalTok{], }\DataTypeTok{use =} \StringTok{"complete.obs"}\NormalTok{)}
\end{Highlighting}
\end{Shaded}

\begin{verbatim}
## [1] -0.143598
\end{verbatim}

Since ``percent time in light'' and ``distance traveled in light'' are
very highly correlated, we'll discard ``distance traveled in light'' and
perform subsequent analyses with only ``percent time in light'' and the
second trait, ``hot plate latency''.

\hypertarget{scatter-plot-of-phenotypes}{%
\subsection{Scatter plot of
phenotypes}\label{scatter-plot-of-phenotypes}}

We create a scatter plot for the two phenotypes, ``hot plate latency''
and ``percent time in light''.

\begin{Shaded}
\begin{Highlighting}[]
\NormalTok{scatter1022 <-}\StringTok{ }\KeywordTok{ggplot}\NormalTok{() }\OperatorTok{+}\StringTok{ }\KeywordTok{geom_point}\NormalTok{(}\DataTypeTok{data =}\NormalTok{ wlph_tib, }\KeywordTok{aes}\NormalTok{(}\DataTypeTok{y =}\NormalTok{ HP_latency, }\DataTypeTok{x =}\NormalTok{ LD_light_pct)) }\OperatorTok{+}\StringTok{ }\KeywordTok{labs}\NormalTok{(}\DataTypeTok{x =} \StringTok{"Percent time in light"}\NormalTok{, }\DataTypeTok{y =} \StringTok{"Hot plate latency"}\NormalTok{) }\OperatorTok{+}\StringTok{ }\KeywordTok{ggtitle}\NormalTok{(}\StringTok{"Scatterplot of hot plate latency vs. percent time in light"}\NormalTok{)}
\NormalTok{scatter1022}
\end{Highlighting}
\end{Shaded}

\begin{verbatim}
## Warning: Removed 3 rows containing missing values (geom_point).
\end{verbatim}

\includegraphics{../../Rpkgs/qtl2pleio/vignettes/recla-analysis_files/figure-latex/unnamed-chunk-10-1.pdf}

\hypertarget{genome-wide-lod-plots-for-the-traits-from-recla}{%
\subsection{Genome-wide LOD plots for the traits from
Recla}\label{genome-wide-lod-plots-for-the-traits-from-recla}}

Let's plot the results of the univariate QTL scans for our two traits.

\begin{Shaded}
\begin{Highlighting}[]
\KeywordTok{plot}\NormalTok{(out, }\DataTypeTok{map =}\NormalTok{ pseudomap, }\DataTypeTok{lodcolumn =} \DecValTok{10}\NormalTok{, }\DataTypeTok{main =} \StringTok{"Genome-wide LOD for percent time in light"}\NormalTok{)}
\end{Highlighting}
\end{Shaded}

\includegraphics{../../Rpkgs/qtl2pleio/vignettes/recla-analysis_files/figure-latex/unnamed-chunk-11-1.pdf}

\begin{Shaded}
\begin{Highlighting}[]
\KeywordTok{plot}\NormalTok{(out, }\DataTypeTok{map =}\NormalTok{ pseudomap, }\DataTypeTok{lodcolumn =} \DecValTok{22}\NormalTok{, }\DataTypeTok{main =} \StringTok{"Genome-wide LOD for hot plate latency"}\NormalTok{)}
\end{Highlighting}
\end{Shaded}

\includegraphics{../../Rpkgs/qtl2pleio/vignettes/recla-analysis_files/figure-latex/unnamed-chunk-12-1.pdf}

\hypertarget{allele-effects-plots-on-chr-8-for-each-of-the-three-recla-traits}{%
\subsection{Allele effects plots on Chr 8 for each of the three Recla
traits}\label{allele-effects-plots-on-chr-8-for-each-of-the-three-recla-traits}}

We examine the allele effects plots for our two traits, in the region of
interest on Chromosome 8.

\begin{Shaded}
\begin{Highlighting}[]
\KeywordTok{scan1coef}\NormalTok{(aprobs[ , }\DecValTok{8}\NormalTok{], }\DataTypeTok{pheno =}\NormalTok{ wlph[, }\DecValTok{10}\NormalTok{], }\DataTypeTok{kinship =}\NormalTok{ kinship}\OperatorTok{$}\StringTok{`}\DataTypeTok{8}\StringTok{`}\NormalTok{, }
          \DataTypeTok{reml =} \OtherTok{TRUE}\NormalTok{,}
          \DataTypeTok{addcovar =}\NormalTok{ sex2) ->}\StringTok{ }\NormalTok{s1c_}\DecValTok{10}
\KeywordTok{scan1coef}\NormalTok{(aprobs[ , }\DecValTok{8}\NormalTok{], }\DataTypeTok{pheno =}\NormalTok{ wlph[, }\DecValTok{22}\NormalTok{], }\DataTypeTok{kinship =}\NormalTok{ kinship}\OperatorTok{$}\StringTok{`}\DataTypeTok{8}\StringTok{`}\NormalTok{, }
          \DataTypeTok{reml =} \OtherTok{TRUE}\NormalTok{,}
          \DataTypeTok{addcovar =}\NormalTok{ sex2) ->}\StringTok{ }\NormalTok{s1c_}\DecValTok{22}
\end{Highlighting}
\end{Shaded}

\begin{Shaded}
\begin{Highlighting}[]
\CommentTok{# subset scan1output objects}
\NormalTok{s1c_10s <-}\StringTok{ }\NormalTok{s1c_}\DecValTok{10}\NormalTok{[}\DecValTok{650}\OperatorTok{:}\DecValTok{999}\NormalTok{, ] }\CommentTok{# 650:999 is the same as the interval for the two-dimensional scan.}
\NormalTok{s1c_22s <-}\StringTok{ }\NormalTok{s1c_}\DecValTok{22}\NormalTok{[}\DecValTok{650}\OperatorTok{:}\DecValTok{999}\NormalTok{, ]}
\end{Highlighting}
\end{Shaded}

\begin{Shaded}
\begin{Highlighting}[]
\KeywordTok{plot_coefCC}\NormalTok{(s1c_10s, }\DataTypeTok{map =}\NormalTok{ pseudomap, }\DataTypeTok{main =} \StringTok{"Allele effects for percent time in light"}\NormalTok{)}
\end{Highlighting}
\end{Shaded}

\includegraphics{../../Rpkgs/qtl2pleio/vignettes/recla-analysis_files/figure-latex/unnamed-chunk-15-1.pdf}

\begin{Shaded}
\begin{Highlighting}[]
\KeywordTok{plot_coefCC}\NormalTok{(s1c_22s, }\DataTypeTok{map =}\NormalTok{ pseudomap, }\DataTypeTok{main =} \StringTok{"Allele effects for hot plate latency"}\NormalTok{)}
\end{Highlighting}
\end{Shaded}

\includegraphics{../../Rpkgs/qtl2pleio/vignettes/recla-analysis_files/figure-latex/unnamed-chunk-15-2.pdf}

\hypertarget{two-dimensional-scan-results-from-github}{%
\subsection{Two-dimensional scan results from
github}\label{two-dimensional-scan-results-from-github}}

We present the code that we ran to perform the two-dimensional scan.

\begin{Shaded}
\begin{Highlighting}[]
\KeywordTok{scan_pvl}\NormalTok{(}\DataTypeTok{probs =}\NormalTok{ pp, }\DataTypeTok{pheno =}\NormalTok{ wlph[, }\KeywordTok{c}\NormalTok{(}\DecValTok{10}\NormalTok{, }\DecValTok{22}\NormalTok{)], }\DataTypeTok{covariates =}\NormalTok{ sex2, }\DataTypeTok{kinship =}\NormalTok{ kinship}\OperatorTok{$}\StringTok{`}\DataTypeTok{8}\StringTok{`}\NormalTok{, }\DataTypeTok{start_snp1 =} \DecValTok{650}\NormalTok{, }\DataTypeTok{n_snp =} \DecValTok{350}\NormalTok{) ->}\StringTok{ }\NormalTok{pvl1022}
\KeywordTok{write.table}\NormalTok{(}\DataTypeTok{x =}\NormalTok{ pvl1022, }\DataTypeTok{file =} \StringTok{"recla-10-22.txt"}\NormalTok{)}
\end{Highlighting}
\end{Shaded}

To save computing time, we read the two-dimensional scan results from
Github.

\begin{Shaded}
\begin{Highlighting}[]
\KeywordTok{as_tibble}\NormalTok{(}\KeywordTok{read.table}\NormalTok{(}\StringTok{"https://raw.githubusercontent.com/fboehm/qtl2pleio-manuscript/master/Rmd/recla-10-22.txt"}\NormalTok{)) ->}\StringTok{ }\NormalTok{pvl1022}
\end{Highlighting}
\end{Shaded}

We then calculate the likelihood ratio test statistic.

\begin{Shaded}
\begin{Highlighting}[]
\NormalTok{(mylrt <-}\StringTok{ }\KeywordTok{calc_lrt_tib}\NormalTok{(pvl1022))}
\end{Highlighting}
\end{Shaded}

\begin{verbatim}
## [1] 2.771408
\end{verbatim}

\hypertarget{profile-lod-plot}{%
\subsection{Profile LOD Plot}\label{profile-lod-plot}}

We create a profile LOD plot.

\begin{Shaded}
\begin{Highlighting}[]
\KeywordTok{colnames}\NormalTok{(recla}\OperatorTok{$}\NormalTok{pheno)[}\KeywordTok{c}\NormalTok{(}\DecValTok{10}\NormalTok{, }\DecValTok{22}\NormalTok{)] <-}\StringTok{ }\KeywordTok{c}\NormalTok{(}\StringTok{"Percent time in light"}\NormalTok{, }\StringTok{"Hot plate latency"}\NormalTok{)}
\NormalTok{p1022 <-}\StringTok{ }\KeywordTok{tidy_scan_pvl}\NormalTok{(pvl1022, }\DataTypeTok{pmap =}\NormalTok{ gm) }\OperatorTok
\StringTok{  }\KeywordTok{add_intercepts}\NormalTok{(}\KeywordTok{c}\NormalTok{(}\KeywordTok{as.numeric}\NormalTok{(pos_LD_light_pct), }\KeywordTok{as.numeric}\NormalTok{(pos_HP_latency))) }\OperatorTok
\StringTok{  }\KeywordTok{plot_pvl}\NormalTok{(}\DataTypeTok{phenames =} \KeywordTok{colnames}\NormalTok{(recla}\OperatorTok{$}\NormalTok{pheno)[}\KeywordTok{c}\NormalTok{(}\DecValTok{10}\NormalTok{, }\DecValTok{22}\NormalTok{)]) }\OperatorTok{+}\StringTok{ }\KeywordTok{ggtitle}\NormalTok{(}\StringTok{"Chromosome 8 profile LOD for percent time in light and hot plate latency"}\NormalTok{) }
\end{Highlighting}
\end{Shaded}

\begin{verbatim}
## Warning: Column `marker`/`marker1` joining character vector and factor,
## coercing into character vector
\end{verbatim}

\begin{verbatim}
## Warning: Column `marker2`/`marker` joining factor and character vector,
## coercing into character vector
\end{verbatim}

\hypertarget{bootstrap-analyses}{%
\subsection{Bootstrap analyses}\label{bootstrap-analyses}}

First, we find the \emph{pleiotropy peak} marker. This is the marker for
which the log likelihood is maximized under the constraint of
pleiotropy.

\begin{Shaded}
\begin{Highlighting}[]
\KeywordTok{find_pleio_peak_tib}\NormalTok{(pvl1022, }\DataTypeTok{start_snp =} \DecValTok{650}\NormalTok{)}
\end{Highlighting}
\end{Shaded}

\begin{verbatim}
## loglik139 
##       788
\end{verbatim}

To save computing time, we read the bootstrap results files from Github.
For details of how we performed the bootstrap analyses on the University
of Wisconsin-Madison Center for High-Throughput Computing, please see
the documentation in the qtl2pleio-manuscript repository:
\url{https://github.com/fboehm/qtl2pleio-manuscript}.

The code below creates a temporary directory ``tmp'' in the user's
working directory. We then download a gzipped tar file that contains
1000 text files. Each text file contains a single likelihood ratio test
statistic from a bootstrap sample.

\begin{Shaded}
\begin{Highlighting}[]
\NormalTok{gz_file <-}\StringTok{ "https://raw.githubusercontent.com/fboehm/qtl2pleio-manuscript/master/chtc/Recla-bootstrap/submit_files/recla-boot-run561.tar.gz"}
\NormalTok{tmp_dir <-}\StringTok{ }\KeywordTok{file.path}\NormalTok{(}\KeywordTok{getwd}\NormalTok{(), }\StringTok{"tmp"}\NormalTok{)}
\KeywordTok{dir.create}\NormalTok{(tmp_dir)}
\KeywordTok{download.file}\NormalTok{(gz_file, }\DataTypeTok{destfile =} \KeywordTok{file.path}\NormalTok{(tmp_dir, }\StringTok{"recla-boot-run561.tar.gz"}\NormalTok{))}
\KeywordTok{untar}\NormalTok{(}\KeywordTok{file.path}\NormalTok{(tmp_dir, }\StringTok{"recla-boot-run561.tar.gz"}\NormalTok{), }\DataTypeTok{exdir =}\NormalTok{ tmp_dir)}
\CommentTok{## read boot lrt files}
\NormalTok{boot_lrt <-}\StringTok{ }\KeywordTok{list}\NormalTok{()}
\ControlFlowTok{for}\NormalTok{ (i }\ControlFlowTok{in} \DecValTok{1}\OperatorTok{:}\DecValTok{1000}\NormalTok{)\{}
\NormalTok{  n <-}\StringTok{ }\NormalTok{i }\OperatorTok{-}\StringTok{ }\DecValTok{1}
\NormalTok{  fn <-}\StringTok{ }\KeywordTok{paste0}\NormalTok{(}\StringTok{"recla-boot-run561_"}\NormalTok{, n, }\StringTok{".txt"}\NormalTok{)}
\NormalTok{  boot_lrt[i] <-}\StringTok{ }\KeywordTok{read.table}\NormalTok{(}\KeywordTok{file.path}\NormalTok{(tmp_dir, fn))}
\NormalTok{\}}
\CommentTok{# convert list to numeric vector}
\NormalTok{boot_lrt <-}\StringTok{ }\KeywordTok{unlist}\NormalTok{(boot_lrt)}
\CommentTok{# delete tmp_dir and its contents}
\KeywordTok{unlink}\NormalTok{(tmp_dir, }\DataTypeTok{recursive =} \OtherTok{TRUE}\NormalTok{)}
\end{Highlighting}
\end{Shaded}

We get a bootstrap p-value by comparing the above vector's values to
\texttt{mylrt}, the test statistic for the observed data.

\begin{Shaded}
\begin{Highlighting}[]
\KeywordTok{sum}\NormalTok{(boot_lrt }\OperatorTok{>=}\StringTok{ }\NormalTok{mylrt) }\OperatorTok{/}\StringTok{ }\KeywordTok{length}\NormalTok{(boot_lrt)}
\end{Highlighting}
\end{Shaded}

\begin{verbatim}
## [1] 0.109
\end{verbatim}

\hypertarget{session-info}{%
\subsection{Session info}\label{session-info}}

\begin{Shaded}
\begin{Highlighting}[]
\NormalTok{devtools}\OperatorTok{::}\KeywordTok{session_info}\NormalTok{()}
\end{Highlighting}
\end{Shaded}

\begin{verbatim}
## - Session info ----------------------------------------------------------
##  setting  value                                      
##  version  R version 3.5.2 Patched (2018-12-24 r75893)
##  os       macOS Mojave 10.14.2                       
##  system   x86_64, darwin15.6.0                       
##  ui       RStudio                                    
##  language (EN)                                       
##  collate  en_US.UTF-8                                
##  ctype    en_US.UTF-8                                
##  tz       America/Chicago                            
##  date     2019-02-05                                 
## 
## - Packages --------------------------------------------------------------
##  package     * version    date       lib source                           
##  assertthat    0.2.0      2017-04-11 [1] CRAN (R 3.5.0)                   
##  backports     1.1.3      2018-12-14 [1] CRAN (R 3.5.0)                   
##  bindr         0.1.1      2018-03-13 [1] CRAN (R 3.5.0)                   
##  bindrcpp    * 0.2.2      2018-03-29 [1] CRAN (R 3.5.0)                   
##  bit           1.1-14     2018-05-29 [1] CRAN (R 3.5.0)                   
##  bit64         0.9-7      2017-05-08 [1] CRAN (R 3.5.0)                   
##  blob          1.1.1      2018-03-25 [1] CRAN (R 3.5.0)                   
##  broman        0.68-2     2018-07-25 [1] CRAN (R 3.5.0)                   
##  callr         3.1.1      2018-12-21 [1] CRAN (R 3.5.0)                   
##  cli           1.0.1      2018-09-25 [1] CRAN (R 3.5.0)                   
##  colorspace    1.3-2      2016-12-14 [1] CRAN (R 3.5.0)                   
##  crayon        1.3.4      2017-09-16 [1] CRAN (R 3.5.0)                   
##  data.table    1.11.8     2018-09-30 [1] CRAN (R 3.5.0)                   
##  DBI           1.0.0      2018-05-02 [1] CRAN (R 3.5.0)                   
##  desc          1.2.0      2018-05-01 [1] CRAN (R 3.5.0)                   
##  devtools      2.0.1      2018-10-26 [1] CRAN (R 3.5.2)                   
##  digest        0.6.18     2018-10-10 [1] CRAN (R 3.5.0)                   
##  dplyr       * 0.7.8      2018-11-10 [1] CRAN (R 3.5.0)                   
##  evaluate      0.12       2018-10-09 [1] CRAN (R 3.5.0)                   
##  fs            1.2.6      2018-08-23 [1] CRAN (R 3.5.0)                   
##  gemma2        0.0.1.1    2018-12-25 [1] Github (fboehm/gemma2@2872396)   
##  ggplot2     * 3.1.0      2018-10-25 [1] CRAN (R 3.5.0)                   
##  glue          1.3.0      2018-07-17 [1] CRAN (R 3.5.0)                   
##  gtable        0.2.0      2016-02-26 [1] CRAN (R 3.5.0)                   
##  htmltools     0.3.6      2017-04-28 [1] CRAN (R 3.5.0)                   
##  jsonlite      1.6        2018-12-07 [1] CRAN (R 3.5.0)                   
##  knitr         1.21       2018-12-10 [1] CRAN (R 3.5.2)                   
##  labeling      0.3        2014-08-23 [1] CRAN (R 3.5.0)                   
##  lattice       0.20-38    2018-11-04 [1] CRAN (R 3.5.2)                   
##  lazyeval      0.2.1      2017-10-29 [1] CRAN (R 3.5.0)                   
##  magrittr      1.5        2014-11-22 [1] CRAN (R 3.5.0)                   
##  MASS          7.3-51.1   2018-11-01 [1] CRAN (R 3.5.2)                   
##  Matrix        1.2-15     2018-11-01 [1] CRAN (R 3.5.2)                   
##  memoise       1.1.0      2017-04-21 [1] CRAN (R 3.5.0)                   
##  munsell       0.5.0      2018-06-12 [1] CRAN (R 3.5.0)                   
##  packrat       0.5.0      2018-11-14 [1] CRAN (R 3.5.0)                   
##  pillar        1.3.1      2018-12-15 [1] CRAN (R 3.5.0)                   
##  pkgbuild      1.0.2      2018-10-16 [1] CRAN (R 3.5.0)                   
##  pkgconfig     2.0.2      2018-08-16 [1] CRAN (R 3.5.0)                   
##  pkgload       1.0.2      2018-10-29 [1] CRAN (R 3.5.0)                   
##  plyr          1.8.4      2016-06-08 [1] CRAN (R 3.5.0)                   
##  prettyunits   1.0.2      2015-07-13 [1] CRAN (R 3.5.0)                   
##  processx      3.2.1      2018-12-05 [1] CRAN (R 3.5.0)                   
##  ps            1.3.0      2018-12-21 [1] CRAN (R 3.5.0)                   
##  purrr         0.2.5      2018-05-29 [1] CRAN (R 3.5.0)                   
##  qtl2        * 0.17-9     2018-12-25 [1] Github (rqtl/qtl2@1c007a2)       
##  qtl2pleio   * 0.1.2.9000 2018-12-31 [1] Github (fboehm/qtl2pleio@be41d66)
##  R6            2.3.0      2018-10-04 [1] CRAN (R 3.5.0)                   
##  Rcpp          1.0.0.1    2018-12-28 [1] Github (RcppCore/Rcpp@0c9f683)   
##  remotes       2.0.2      2018-10-30 [1] CRAN (R 3.5.0)                   
##  rlang         0.3.0.1    2018-10-25 [1] CRAN (R 3.5.0)                   
##  rmarkdown   * 1.11       2018-12-08 [1] CRAN (R 3.5.0)                   
##  rprojroot     1.3-2      2018-01-03 [1] CRAN (R 3.5.0)                   
##  RSQLite       2.1.1      2018-05-06 [1] CRAN (R 3.5.0)                   
##  scales        1.0.0      2018-08-09 [1] CRAN (R 3.5.0)                   
##  sessioninfo   1.1.1      2018-11-05 [1] CRAN (R 3.5.0)                   
##  stringi       1.2.4      2018-07-20 [1] CRAN (R 3.5.0)                   
##  stringr       1.3.1      2018-05-10 [1] CRAN (R 3.5.0)                   
##  testthat      2.0.1      2018-10-13 [1] CRAN (R 3.5.0)                   
##  tibble        1.4.2      2018-01-22 [1] CRAN (R 3.5.0)                   
##  tidyselect    0.2.5      2018-10-11 [1] CRAN (R 3.5.0)                   
##  tinytex       0.9        2018-10-23 [1] CRAN (R 3.5.0)                   
##  usethis       1.4.0      2018-08-14 [1] CRAN (R 3.5.0)                   
##  withr         2.1.2      2018-03-15 [1] CRAN (R 3.5.0)                   
##  xfun          0.4        2018-10-23 [1] CRAN (R 3.5.0)                   
##  yaml          2.2.0      2018-07-25 [1] CRAN (R 3.5.0)                   
## 
## [1] /Library/Frameworks/R.framework/Versions/3.5/Resources/library
\end{verbatim}

\section{Pleiotropy testing vignette}

\hypertarget{setting-for-pleiotropy-vs.two-separate-qtl}{%
\subsection{Setting for pleiotropy vs.~two separate
QTL}\label{setting-for-pleiotropy-vs.two-separate-qtl}}

Our setting involves a pair of traits, \(Y_1\) and \(Y_2\), each of
which individually (univariately) maps to a single genomic region.
\(Y_1\) and \(Y_2\) are both measured on the same subjects. The exact
definition of a genomic region is imprecise; in practice, it may be as
large as 4 or 5 Mb. We seek to distinguish whether \(Y_1\) and \(Y_2\)
associations (in the genomic region of interest) arise due to a single
QTL or whether there are two two distinct loci, each of which associates
with exactly one of the two traits. We recognize that more complicated
association patterns are possible, but we neglect them in this test.

\hypertarget{installing-qtl2pleio}{%
\subsection{\texorpdfstring{Installing
\texttt{qtl2pleio}}{Installing qtl2pleio}}\label{installing-qtl2pleio}}

We install \texttt{qtl2pleio} from github via the \texttt{devtools} R
package, which is available from CRAN.

If you haven't installed \texttt{devtools} R package, you can do so with
this line of code:

\begin{Shaded}
\begin{Highlighting}[]
\KeywordTok{install.packages}\NormalTok{(}\StringTok{"devtools"}\NormalTok{)}
\end{Highlighting}
\end{Shaded}

Now that you have installed \texttt{devtools}, you can install
\texttt{qtl2pleio} from its Github repository with this line of code:

\begin{Shaded}
\begin{Highlighting}[]
\NormalTok{devtools}\OperatorTok{::}\KeywordTok{install_github}\NormalTok{(}\StringTok{"fboehm/qtl2pleio"}\NormalTok{)}
\end{Highlighting}
\end{Shaded}

The above line only needs to be run once on a given computer (unless you
wish to install a newer version of the package).

We then load the library into our R session with the \texttt{library}
command:

\begin{Shaded}
\begin{Highlighting}[]
\KeywordTok{library}\NormalTok{(qtl2pleio)}
\end{Highlighting}
\end{Shaded}

We'll work with data from the \texttt{qtl2data} repository, which is on
github. First, we install and load the \texttt{qtl2} package.

\begin{Shaded}
\begin{Highlighting}[]
\NormalTok{devtools}\OperatorTok{::}\KeywordTok{install_github}\NormalTok{(}\StringTok{"rqtl/qtl2"}\NormalTok{)}
\end{Highlighting}
\end{Shaded}

We use the above line once to install the package on our computer before
loading the package with the \texttt{library} command.

\begin{Shaded}
\begin{Highlighting}[]
\KeywordTok{library}\NormalTok{(qtl2)}
\end{Highlighting}
\end{Shaded}

\hypertarget{reading-data-from-qtl2data-repository-on-github}{%
\subsection{\texorpdfstring{Reading data from \texttt{qtl2data}
repository on
github}{Reading data from qtl2data repository on github}}\label{reading-data-from-qtl2data-repository-on-github}}

We read from github.com data from the \texttt{qtl2data} repository.

\begin{Shaded}
\begin{Highlighting}[]
\NormalTok{tmpfile <-}\StringTok{ }\KeywordTok{tempfile}\NormalTok{()}
\NormalTok{file <-}\StringTok{ }\KeywordTok{paste0}\NormalTok{(}\StringTok{"https://raw.githubusercontent.com/rqtl/"}\NormalTok{,}
               \StringTok{"qtl2data/master/DOex/DOex_alleleprobs.rds"}\NormalTok{)}
\KeywordTok{download.file}\NormalTok{(file, tmpfile)}
\NormalTok{pr <-}\StringTok{ }\KeywordTok{readRDS}\NormalTok{(tmpfile)}
\KeywordTok{unlink}\NormalTok{(tmpfile)}
\NormalTok{tmpfile <-}\StringTok{ }\KeywordTok{tempfile}\NormalTok{()}
\NormalTok{file <-}\StringTok{ }\KeywordTok{paste0}\NormalTok{(}\StringTok{"https://raw.githubusercontent.com/rqtl/"}\NormalTok{,}
               \StringTok{"qtl2data/master/DOex/DOex_pmap.csv"}\NormalTok{)}
\KeywordTok{download.file}\NormalTok{(file, tmpfile)}
\NormalTok{pmap_pre <-}\StringTok{ }\KeywordTok{read.csv}\NormalTok{(tmpfile)}
\KeywordTok{unlink}\NormalTok{(tmpfile)}
\NormalTok{pm2 <-}\StringTok{ }\NormalTok{pmap_pre[pmap_pre}\OperatorTok{$}\NormalTok{chr }\OperatorTok{==}\StringTok{ }\DecValTok{2}\NormalTok{, }\DecValTok{3}\NormalTok{]}
\KeywordTok{names}\NormalTok{(pm2) <-}\StringTok{ }\NormalTok{pmap_pre[pmap_pre}\OperatorTok{$}\NormalTok{chr }\OperatorTok{==}\StringTok{ }\DecValTok{2}\NormalTok{, }\DecValTok{1}\NormalTok{]}
\NormalTok{pm3 <-}\StringTok{ }\NormalTok{pmap_pre[pmap_pre}\OperatorTok{$}\NormalTok{chr }\OperatorTok{==}\StringTok{ }\DecValTok{3}\NormalTok{, }\DecValTok{3}\NormalTok{]}
\KeywordTok{names}\NormalTok{(pm3) <-}\StringTok{ }\NormalTok{pmap_pre[pmap_pre}\OperatorTok{$}\NormalTok{chr }\OperatorTok{==}\StringTok{ }\DecValTok{3}\NormalTok{, }\DecValTok{1}\NormalTok{]}
\NormalTok{pmX <-}\StringTok{ }\NormalTok{pmap_pre[pmap_pre}\OperatorTok{$}\NormalTok{chr }\OperatorTok{==}\StringTok{ "X"}\NormalTok{, }\DecValTok{3}\NormalTok{]}
\KeywordTok{names}\NormalTok{(pmX) <-}\StringTok{ }\NormalTok{pmap_pre[pmap_pre}\OperatorTok{$}\NormalTok{chr }\OperatorTok{==}\StringTok{ "X"}\NormalTok{, }\DecValTok{1}\NormalTok{]}
\KeywordTok{list}\NormalTok{(pm2, pm3, pmX) ->}\StringTok{ }\NormalTok{pm}
\KeywordTok{names}\NormalTok{(pm) <-}\StringTok{ }\KeywordTok{c}\NormalTok{(}\StringTok{"`2`"}\NormalTok{, }\StringTok{"`3`"}\NormalTok{, }\StringTok{"X"}\NormalTok{)}
\end{Highlighting}
\end{Shaded}

We now have an allele probabilities object stored in \texttt{pr}.

\begin{Shaded}
\begin{Highlighting}[]
\KeywordTok{names}\NormalTok{(pr)}
\CommentTok{#> [1] "2" "3" "X"}
\KeywordTok{dim}\NormalTok{(pr}\OperatorTok{$}\StringTok{`}\DataTypeTok{2}\StringTok{`}\NormalTok{)}
\CommentTok{#> [1] 261   8 127}
\end{Highlighting}
\end{Shaded}

We see that \texttt{pr} is a list of 3 three-dimensional arrays - one
array for each of 3 chromosomes.

\hypertarget{kinship-calculations}{%
\subsection{Kinship calculations}\label{kinship-calculations}}

For our statistical model, we need a kinship matrix. Although we don't
have genome-wide data - since we have allele probabilities for only 3
chromosomes - let's calculate a kinship matrix using
``leave-one-chromosome-out''. In practice, one would want to use allele
probabilities from a full genome-wide set of markers.

\begin{Shaded}
\begin{Highlighting}[]
\KeywordTok{calc_kinship}\NormalTok{(}\DataTypeTok{probs =}\NormalTok{ pr, }\DataTypeTok{type =} \StringTok{"loco"}\NormalTok{) ->}\StringTok{ }\NormalTok{kinship}
\end{Highlighting}
\end{Shaded}

\begin{Shaded}
\begin{Highlighting}[]
\KeywordTok{str}\NormalTok{(kinship)}
\CommentTok{#> List of 3}
\CommentTok{#>  $ 2: num [1:261, 1:261] 0.6919 0.0707 0.2355 0.0558 0.0512 ...}
\CommentTok{#>   ..- attr(*, "n_pos")= int 195}
\CommentTok{#>   ..- attr(*, "dimnames")=List of 2}
\CommentTok{#>   .. ..$ : chr [1:261] "1" "4" "5" "6" ...}
\CommentTok{#>   .. ..$ : chr [1:261] "1" "4" "5" "6" ...}
\CommentTok{#>  $ 3: num [1:261, 1:261] 0.6642 0.0638 0.2035 0.1128 0.0772 ...}
\CommentTok{#>   ..- attr(*, "n_pos")= int 220}
\CommentTok{#>   ..- attr(*, "dimnames")=List of 2}
\CommentTok{#>   .. ..$ : chr [1:261] "1" "4" "5" "6" ...}
\CommentTok{#>   .. ..$ : chr [1:261] "1" "4" "5" "6" ...}
\CommentTok{#>  $ X: num [1:261, 1:261] 0.4853 0.0821 0.1958 0.1039 0.1123 ...}
\CommentTok{#>   ..- attr(*, "n_pos")= int 229}
\CommentTok{#>   ..- attr(*, "dimnames")=List of 2}
\CommentTok{#>   .. ..$ : chr [1:261] "1" "4" "5" "6" ...}
\CommentTok{#>   .. ..$ : chr [1:261] "1" "4" "5" "6" ...}
\end{Highlighting}
\end{Shaded}

We see that \texttt{kinship} is a list containing 3 matrices. Each
matrix is 261 by 261 - where the number of subjects is 261 - and
symmetric. The \((i, j)\) cell in the matrix contains the estimate of
identity-by-state (IBS) probability for randomly chosen alleles at a
single locus for those two subjects.

Before we simulate phenotype data, we first specify our statistical
model.

We use the model:

\[vec(Y) = X vec(B) + vec(G) + vec(E)\]

where \(Y\) is a \(n\) by \(2\) matrix, where each row is one subject
and each column is one quantitative trait. \(X\) is a \(2n\) by \(2f\)
design matrix containing \(n\) by \(f\) allele probabilities matrices
for each of two (possibly identical) markers. Thus, \(X\) is a
block-diagonal matrix, with exactly two \(n\) by \(f\) blocks on the
diagonal. \(B\) is a \(f\) by 2 matrix. ``vec'' refers to the
vectorization operator. ``vec(B)'', where \(B\) is a \(f\) by \(2\)
matrix, is, thus, a (column) vector of length \(2f\) that is formed by
stacking the second column of \(B\) beneath the first column of \(B\).

\(G\) is a matrix of random effects. We specify its distribution as
matrix-variate normal with mean being a \(n\) by \(2\) matrix of zeros,
covariance among row vectors a \(n\) by \(n\) kinship matrix, \(K\), and
covariance among column vectors a \(2\) by \(2\) genetic covariance
matrix, \(V_g\).

In mathematical notation, we write:

\[G \sim MN_{\text{n by 2}}(0, K, V_g)\]

We also need to specify the distribution of the \(E\) matrix, which
contains the random errors. \(E\) is a random \(n\) by \(2\) matrix that
is distributed as a matrix-variate normal distribution with mean being
the \(n\) by \(2\) zero matrix, covariance among row vectors \(I_n\),
the \(n\) by \(n\) identity matrix, and covariance among columns the
\(2\) by \(2\) matrix \(V_e\).

\[E \sim MN_{\text{n by 2}}(0, I_n, V_e)\]

In practice, we typically observe the phenotype matrix \(Y\). We also
treat as known the design matrix \(X\) and the kinship matrix \(K\). We
then infer the values of \(B\), \(V_g\), and \(V_e\).

\hypertarget{simulating-phenotypes-with-qtl2pleiosim1}{%
\subsection{\texorpdfstring{Simulating phenotypes with
\texttt{qtl2pleio::sim1}}{Simulating phenotypes with qtl2pleio::sim1}}\label{simulating-phenotypes-with-qtl2pleiosim1}}

The function to simulate phenotypes in \texttt{qtl2pleio} is
\texttt{sim1}. By examining its help page, we see that it takes five
arguments. The help page also gives the dimensions of the inputs.

\begin{Shaded}
\begin{Highlighting}[]
\CommentTok{# set up the design matrix, X}
\NormalTok{pp <-}\StringTok{ }\NormalTok{pr[[}\DecValTok{2}\NormalTok{]]}
\NormalTok{X <-}\StringTok{ }\NormalTok{gemma2}\OperatorTok{::}\KeywordTok{stagger_mats}\NormalTok{(pp[ , , }\DecValTok{50}\NormalTok{], pp[ , , }\DecValTok{50}\NormalTok{])}
\CommentTok{# assemble B matrix of allele effects}
\NormalTok{B <-}\StringTok{ }\KeywordTok{matrix}\NormalTok{(}\DataTypeTok{data =} \KeywordTok{c}\NormalTok{(}\OperatorTok{-}\DecValTok{1}\NormalTok{, }\DecValTok{-1}\NormalTok{, }\DecValTok{-1}\NormalTok{, }\DecValTok{-1}\NormalTok{, }\DecValTok{1}\NormalTok{, }\DecValTok{1}\NormalTok{, }\DecValTok{1}\NormalTok{, }\DecValTok{1}\NormalTok{, }\DecValTok{-1}\NormalTok{, }\DecValTok{-1}\NormalTok{, }\DecValTok{-1}\NormalTok{, }\DecValTok{-1}\NormalTok{, }\DecValTok{1}\NormalTok{, }\DecValTok{1}\NormalTok{, }\DecValTok{1}\NormalTok{, }\DecValTok{1}\NormalTok{), }\DataTypeTok{nrow =} \DecValTok{8}\NormalTok{, }\DataTypeTok{ncol =} \DecValTok{2}\NormalTok{, }\DataTypeTok{byrow =} \OtherTok{FALSE}\NormalTok{)}
\CommentTok{# verify that B is what we want:}
\NormalTok{B}
\CommentTok{#>      [,1] [,2]}
\CommentTok{#> [1,]   -1   -1}
\CommentTok{#> [2,]   -1   -1}
\CommentTok{#> [3,]   -1   -1}
\CommentTok{#> [4,]   -1   -1}
\CommentTok{#> [5,]    1    1}
\CommentTok{#> [6,]    1    1}
\CommentTok{#> [7,]    1    1}
\CommentTok{#> [8,]    1    1}
\CommentTok{# set.seed to ensure reproducibility}
\KeywordTok{set.seed}\NormalTok{(}\DecValTok{2018-01-30}\NormalTok{)}
\CommentTok{# call to sim1}
\NormalTok{Ypre <-}\StringTok{ }\KeywordTok{sim1}\NormalTok{(}\DataTypeTok{X =}\NormalTok{ X, }\DataTypeTok{B =}\NormalTok{ B, }\DataTypeTok{Vg =} \KeywordTok{diag}\NormalTok{(}\DecValTok{2}\NormalTok{), }\DataTypeTok{Ve =} \KeywordTok{diag}\NormalTok{(}\DecValTok{2}\NormalTok{), }\DataTypeTok{kinship =}\NormalTok{ kinship[[}\DecValTok{2}\NormalTok{]])}
\NormalTok{Y <-}\StringTok{ }\KeywordTok{matrix}\NormalTok{(Ypre, }\DataTypeTok{nrow =} \DecValTok{261}\NormalTok{, }\DataTypeTok{ncol =} \DecValTok{2}\NormalTok{, }\DataTypeTok{byrow =} \OtherTok{FALSE}\NormalTok{)}
\KeywordTok{rownames}\NormalTok{(Y) <-}\StringTok{ }\KeywordTok{rownames}\NormalTok{(pp)}
\KeywordTok{colnames}\NormalTok{(Y) <-}\StringTok{ }\KeywordTok{c}\NormalTok{(}\StringTok{"tr1"}\NormalTok{, }\StringTok{"tr2"}\NormalTok{)}
\end{Highlighting}
\end{Shaded}

Let's perform univariate QTL mapping for each of the two traits in the Y
matrix.

\begin{Shaded}
\begin{Highlighting}[]
\KeywordTok{scan1}\NormalTok{(}\DataTypeTok{genoprobs =}\NormalTok{ pr, }\DataTypeTok{pheno =}\NormalTok{ Y, }\DataTypeTok{kinship =}\NormalTok{ kinship) ->}\StringTok{ }\NormalTok{s1}
\end{Highlighting}
\end{Shaded}

\begin{Shaded}
\begin{Highlighting}[]
\KeywordTok{plot}\NormalTok{(s1, pm)}
\end{Highlighting}
\end{Shaded}

\includegraphics{../../Rpkgs/qtl2pleio/vignettes/testing-pleiotropy-vs-separate-qtl_files/figure-latex/unnamed-chunk-12-1.pdf}

\begin{Shaded}
\begin{Highlighting}[]
\KeywordTok{find_peaks}\NormalTok{(s1, }\DataTypeTok{map =}\NormalTok{ pm)}
\CommentTok{#>   lodindex lodcolumn chr       pos       lod}
\CommentTok{#> 1        1       tr1 `2` 75.562440  3.636498}
\CommentTok{#> 2        1       tr1 `3` 82.778059 24.552172}
\CommentTok{#> 3        1       tr1   X 13.406552  3.585519}
\CommentTok{#> 4        2       tr2 `2`  3.164247  8.195879}
\CommentTok{#> 5        2       tr2 `3` 82.778059 17.454825}
\CommentTok{#> 6        2       tr2   X 95.069663 14.110670}
\end{Highlighting}
\end{Shaded}

\hypertarget{perform-two-dimensional-scan-as-first-step-in-pleiotropy-v-separate-qtl-hypothesis-test}{%
\section{Perform two-dimensional scan as first step in pleiotropy v
separate QTL hypothesis
test}\label{perform-two-dimensional-scan-as-first-step-in-pleiotropy-v-separate-qtl-hypothesis-test}}

\begin{Shaded}
\begin{Highlighting}[]
\NormalTok{out <-}\StringTok{ }\KeywordTok{scan_pvl}\NormalTok{(}\DataTypeTok{probs =}\NormalTok{ pp, }
                \DataTypeTok{pheno =}\NormalTok{ Y, }
                \DataTypeTok{kinship =}\NormalTok{ kinship[[}\DecValTok{2}\NormalTok{]], }\CommentTok{# 2nd entry in kinship list is Chr 3}
                \DataTypeTok{start_snp =} \DecValTok{38}\NormalTok{, }
                \DataTypeTok{n_snp =} \DecValTok{25}\NormalTok{, }\DataTypeTok{n_cores =} \DecValTok{1}
\NormalTok{                )}
\CommentTok{#> starting covariance matrices estimation with data from 261 subjects.}
\CommentTok{#> covariance matrices estimation completed.}
\end{Highlighting}
\end{Shaded}

\hypertarget{create-a-profile-lod-plot-to-visualize-results-of-two-dimensional-scan}{%
\subsubsection{Create a profile LOD plot to visualize results of
two-dimensional
scan}\label{create-a-profile-lod-plot-to-visualize-results-of-two-dimensional-scan}}

\begin{Shaded}
\begin{Highlighting}[]
\KeywordTok{library}\NormalTok{(dplyr)}
\NormalTok{out }\OperatorTok\StringTok{ }
\StringTok{  }\KeywordTok{tidy_scan_pvl}\NormalTok{(pm3) }\OperatorTok\StringTok{ }\CommentTok{# pm3 is physical map for Chr 3}
\StringTok{  }\KeywordTok{add_intercepts}\NormalTok{(}\DataTypeTok{intercepts_univariate =} \KeywordTok{c}\NormalTok{(}\FloatTok{82.8}\NormalTok{, }\FloatTok{82.8}\NormalTok{)) }\OperatorTok
\StringTok{  }\KeywordTok{plot_pvl}\NormalTok{(}\DataTypeTok{phenames =} \KeywordTok{c}\NormalTok{(}\StringTok{"tr1"}\NormalTok{, }\StringTok{"tr2"}\NormalTok{))}
\CommentTok{#> Warning: Removed 50 rows containing missing values (geom_path).}
\end{Highlighting}
\end{Shaded}

\includegraphics{../../Rpkgs/qtl2pleio/vignettes/testing-pleiotropy-vs-separate-qtl_files/figure-latex/unnamed-chunk-15-1.pdf}

\hypertarget{calculate-the-likelihood-ratio-test-statistic-for-pleiotropy-v-separate-qtl}{%
\subsubsection{Calculate the likelihood ratio test statistic for
pleiotropy v separate
QTL}\label{calculate-the-likelihood-ratio-test-statistic-for-pleiotropy-v-separate-qtl}}

We use the function \texttt{calc\_lrt\_tib} to calculate the likelihood
ratio test statistic value for the specified traits and specified
genomic region.

\begin{Shaded}
\begin{Highlighting}[]
\NormalTok{(}\KeywordTok{calc_lrt_tib}\NormalTok{(out) ->}\StringTok{ }\NormalTok{lrt)}
\CommentTok{#> [1] 0.7730965}
\end{Highlighting}
\end{Shaded}

\hypertarget{bootstrap-analysis-to-get-p-values}{%
\subsection{Bootstrap analysis to get
p-values}\label{bootstrap-analysis-to-get-p-values}}

The calibration of test statistic values to get p-values uses bootstrap
methods because we don't know the theoretical distribution of the test
statistic under the null hypothesis. Thus, we use a bootstrap approach
to obtain an empirical distribution of test statistic values under the
null hypothesis of the presence of one pleiotropic locus.

We will use the function \texttt{boot\_pvl} from our package
\texttt{qtl2pleio}.

We use a parametric bootstrap strategy in which we first use the studied
phenotypes to infer the values of model parameters. Once we have the
inferred values of the model parameters, we simulate phenotypes from the
pleiotropy model (with the inferred parameter values).

A natural question that arises is ``which marker's allele probabilities
do we use when simulating phenotypes?'' We use the marker that, under
the null hypothesis, ie, under the pleiotropy constraint, yields the
greatest value of the log-likelihood.

Before we call \texttt{boot\_pvl}, we need to identify the index (on the
chromosome under study) of the marker that maximizes the likelihood
under the pleiotropy constraint. To do this, we use the
\texttt{qtl2pleio} function \texttt{find\_pleio\_peak\_tib}.

\begin{Shaded}
\begin{Highlighting}[]
\NormalTok{(pleio_index <-}\StringTok{ }\KeywordTok{find_pleio_peak_tib}\NormalTok{(out, }\DataTypeTok{start_snp =} \DecValTok{38}\NormalTok{))}
\CommentTok{#> loglik13 }
\CommentTok{#>       50}
\end{Highlighting}
\end{Shaded}

\begin{Shaded}
\begin{Highlighting}[]
\KeywordTok{set.seed}\NormalTok{(}\DecValTok{2018-10-03}\NormalTok{)}
\KeywordTok{system.time}\NormalTok{(b_out <-}\StringTok{ }\KeywordTok{boot_pvl}\NormalTok{(}\DataTypeTok{probs =}\NormalTok{ pp,}
         \DataTypeTok{pheno =}\NormalTok{ Y, }
         \DataTypeTok{pleio_peak_index =}\NormalTok{ pleio_index, }
         \DataTypeTok{kinship =}\NormalTok{ kinship[[}\DecValTok{2}\NormalTok{]], }\CommentTok{# 2nd element in kinship list is Chr 3 }
         \DataTypeTok{nboot_per_job =} \DecValTok{10}\NormalTok{, }
         \DataTypeTok{start_snp =} \DecValTok{38}\NormalTok{, }
         \DataTypeTok{n_snp =} \DecValTok{10}
\NormalTok{         ))}
\CommentTok{#> starting covariance matrices estimation with data from 261 subjects.}
\CommentTok{#> covariance matrices estimation completed.}
\CommentTok{#> starting covariance matrices estimation with data from 261 subjects.}
\CommentTok{#> covariance matrices estimation completed.}
\CommentTok{#> starting covariance matrices estimation with data from 261 subjects.}
\CommentTok{#> covariance matrices estimation completed.}
\CommentTok{#> starting covariance matrices estimation with data from 261 subjects.}
\CommentTok{#> covariance matrices estimation completed.}
\CommentTok{#> starting covariance matrices estimation with data from 261 subjects.}
\CommentTok{#> covariance matrices estimation completed.}
\CommentTok{#> starting covariance matrices estimation with data from 261 subjects.}
\CommentTok{#> covariance matrices estimation completed.}
\CommentTok{#> starting covariance matrices estimation with data from 261 subjects.}
\CommentTok{#> covariance matrices estimation completed.}
\CommentTok{#> starting covariance matrices estimation with data from 261 subjects.}
\CommentTok{#> covariance matrices estimation completed.}
\CommentTok{#> starting covariance matrices estimation with data from 261 subjects.}
\CommentTok{#> covariance matrices estimation completed.}
\CommentTok{#> starting covariance matrices estimation with data from 261 subjects.}
\CommentTok{#> covariance matrices estimation completed.}
\CommentTok{#>    user  system elapsed }
\CommentTok{#>  53.802   0.375  54.249}
\end{Highlighting}
\end{Shaded}

The argument \texttt{nboot\_per\_job} indicates the number of bootstrap
samples that will be created and analyzed. Here, we set
\texttt{nboot\_per\_job\ =\ 10}, so we expect to see returned a numeric
vector of length 10, where each entry is a LRT statistic value from a
distinct bootstrap sample.

Finally, we determine a bootstrap p-value in the usual method. We treat
the bootstrap samples' test statistics as an empirical distribution of
the test statistic under the null hypothesis of pleiotropy. Thus, to get
a p-value, we want to ask ``What is the probability, under the null
hypothesis, of observing a test statistic value that is at least as
extreme as that which we observed?''

\begin{Shaded}
\begin{Highlighting}[]
\NormalTok{(pvalue <-}\StringTok{ }\KeywordTok{mean}\NormalTok{(b_out }\OperatorTok{>=}\StringTok{ }\NormalTok{lrt))}
\CommentTok{#> [1] 0.1}
\end{Highlighting}
\end{Shaded}

Typically, one would want to conduct the two-dimensional scan over a
larger grid (ie, with more markers). We abbreviated the scan above to
reduce computing time.

In practice, one would want to use many more bootstrap samples to
achieve an empirical distribution that is closer to the theoretical
distribution of the test statistic under the null hypothesis.

However, if one wants to perform analyses with a reasonable number - say
400 - bootstrap samples, this will take a very long time - several days
- on a single laptop computer. We have used a series of computer
clusters that are coordinated by the University of Wisconsin-Madison's
Center for High-throughput Computing (\url{http://chtc.cs.wisc.edu}). We
typically are able to analyze 1000 bootstrap samples in less than 24
hours with this service.

We recently added parallel computing capacity with \texttt{boot\_pvl}
and \texttt{scan\_pvl}. The user may specify \texttt{n\_cores} when
calling each function.

\hypertarget{session-info}{%
\subsection{Session info}\label{session-info-2}}

\begin{Shaded}
\begin{Highlighting}[]
\NormalTok{devtools}\OperatorTok{::}\KeywordTok{session_info}\NormalTok{()}
\CommentTok{#> - Session info ----------------------------------------------------------}
\CommentTok{#>  setting  value                                      }
\CommentTok{#>  version  R version 3.5.2 Patched (2018-12-24 r75893)}
\CommentTok{#>  os       macOS Mojave 10.14.2                       }
\CommentTok{#>  system   x86_64, darwin15.6.0                       }
\CommentTok{#>  ui       RStudio                                    }
\CommentTok{#>  language (EN)                                       }
\CommentTok{#>  collate  en_US.UTF-8                                }
\CommentTok{#>  ctype    en_US.UTF-8                                }
\CommentTok{#>  tz       America/Chicago                            }
\CommentTok{#>  date     2019-02-05                                 }
\CommentTok{#> }
\CommentTok{#> - Packages --------------------------------------------------------------}
\CommentTok{#>  package     * version    date       lib source                           }
\CommentTok{#>  assertthat    0.2.0      2017-04-11 [1] CRAN (R 3.5.0)                   }
\CommentTok{#>  backports     1.1.3      2018-12-14 [1] CRAN (R 3.5.0)                   }
\CommentTok{#>  bindr         0.1.1      2018-03-13 [1] CRAN (R 3.5.0)                   }
\CommentTok{#>  bindrcpp    * 0.2.2      2018-03-29 [1] CRAN (R 3.5.0)                   }
\CommentTok{#>  bit           1.1-14     2018-05-29 [1] CRAN (R 3.5.0)                   }
\CommentTok{#>  bit64         0.9-7      2017-05-08 [1] CRAN (R 3.5.0)                   }
\CommentTok{#>  blob          1.1.1      2018-03-25 [1] CRAN (R 3.5.0)                   }
\CommentTok{#>  broman        0.68-2     2018-07-25 [1] CRAN (R 3.5.0)                   }
\CommentTok{#>  callr         3.1.1      2018-12-21 [1] CRAN (R 3.5.0)                   }
\CommentTok{#>  cli           1.0.1      2018-09-25 [1] CRAN (R 3.5.0)                   }
\CommentTok{#>  colorspace    1.3-2      2016-12-14 [1] CRAN (R 3.5.0)                   }
\CommentTok{#>  crayon        1.3.4      2017-09-16 [1] CRAN (R 3.5.0)                   }
\CommentTok{#>  data.table    1.11.8     2018-09-30 [1] CRAN (R 3.5.0)                   }
\CommentTok{#>  DBI           1.0.0      2018-05-02 [1] CRAN (R 3.5.0)                   }
\CommentTok{#>  desc          1.2.0      2018-05-01 [1] CRAN (R 3.5.0)                   }
\CommentTok{#>  devtools      2.0.1      2018-10-26 [1] CRAN (R 3.5.2)                   }
\CommentTok{#>  digest        0.6.18     2018-10-10 [1] CRAN (R 3.5.0)                   }
\CommentTok{#>  dplyr       * 0.7.8      2018-11-10 [1] CRAN (R 3.5.0)                   }
\CommentTok{#>  evaluate      0.12       2018-10-09 [1] CRAN (R 3.5.0)                   }
\CommentTok{#>  fs            1.2.6      2018-08-23 [1] CRAN (R 3.5.0)                   }
\CommentTok{#>  gemma2        0.0.1.1    2018-12-25 [1] Github (fboehm/gemma2@2872396)   }
\CommentTok{#>  ggplot2     * 3.1.0      2018-10-25 [1] CRAN (R 3.5.0)                   }
\CommentTok{#>  glue          1.3.0      2018-07-17 [1] CRAN (R 3.5.0)                   }
\CommentTok{#>  gtable        0.2.0      2016-02-26 [1] CRAN (R 3.5.0)                   }
\CommentTok{#>  htmltools     0.3.6      2017-04-28 [1] CRAN (R 3.5.0)                   }
\CommentTok{#>  jsonlite      1.6        2018-12-07 [1] CRAN (R 3.5.0)                   }
\CommentTok{#>  knitr         1.21       2018-12-10 [1] CRAN (R 3.5.2)                   }
\CommentTok{#>  labeling      0.3        2014-08-23 [1] CRAN (R 3.5.0)                   }
\CommentTok{#>  lattice       0.20-38    2018-11-04 [1] CRAN (R 3.5.2)                   }
\CommentTok{#>  lazyeval      0.2.1      2017-10-29 [1] CRAN (R 3.5.0)                   }
\CommentTok{#>  magrittr      1.5        2014-11-22 [1] CRAN (R 3.5.0)                   }
\CommentTok{#>  MASS          7.3-51.1   2018-11-01 [1] CRAN (R 3.5.2)                   }
\CommentTok{#>  Matrix        1.2-15     2018-11-01 [1] CRAN (R 3.5.2)                   }
\CommentTok{#>  memoise       1.1.0      2017-04-21 [1] CRAN (R 3.5.0)                   }
\CommentTok{#>  munsell       0.5.0      2018-06-12 [1] CRAN (R 3.5.0)                   }
\CommentTok{#>  packrat       0.5.0      2018-11-14 [1] CRAN (R 3.5.0)                   }
\CommentTok{#>  pillar        1.3.1      2018-12-15 [1] CRAN (R 3.5.0)                   }
\CommentTok{#>  pkgbuild      1.0.2      2018-10-16 [1] CRAN (R 3.5.0)                   }
\CommentTok{#>  pkgconfig     2.0.2      2018-08-16 [1] CRAN (R 3.5.0)                   }
\CommentTok{#>  pkgload       1.0.2      2018-10-29 [1] CRAN (R 3.5.0)                   }
\CommentTok{#>  plyr          1.8.4      2016-06-08 [1] CRAN (R 3.5.0)                   }
\CommentTok{#>  prettyunits   1.0.2      2015-07-13 [1] CRAN (R 3.5.0)                   }
\CommentTok{#>  processx      3.2.1      2018-12-05 [1] CRAN (R 3.5.0)                   }
\CommentTok{#>  ps            1.3.0      2018-12-21 [1] CRAN (R 3.5.0)                   }
\CommentTok{#>  purrr         0.2.5      2018-05-29 [1] CRAN (R 3.5.0)                   }
\CommentTok{#>  qtl2        * 0.17-9     2018-12-25 [1] Github (rqtl/qtl2@1c007a2)       }
\CommentTok{#>  qtl2pleio   * 0.1.2.9000 2018-12-31 [1] Github (fboehm/qtl2pleio@be41d66)}
\CommentTok{#>  R6            2.3.0      2018-10-04 [1] CRAN (R 3.5.0)                   }
\CommentTok{#>  Rcpp          1.0.0.1    2018-12-28 [1] Github (RcppCore/Rcpp@0c9f683)   }
\CommentTok{#>  remotes       2.0.2      2018-10-30 [1] CRAN (R 3.5.0)                   }
\CommentTok{#>  rlang         0.3.0.1    2018-10-25 [1] CRAN (R 3.5.0)                   }
\CommentTok{#>  rmarkdown   * 1.11       2018-12-08 [1] CRAN (R 3.5.0)                   }
\CommentTok{#>  rprojroot     1.3-2      2018-01-03 [1] CRAN (R 3.5.0)                   }
\CommentTok{#>  RSQLite       2.1.1      2018-05-06 [1] CRAN (R 3.5.0)                   }
\CommentTok{#>  scales        1.0.0      2018-08-09 [1] CRAN (R 3.5.0)                   }
\CommentTok{#>  sessioninfo   1.1.1      2018-11-05 [1] CRAN (R 3.5.0)                   }
\CommentTok{#>  stringi       1.2.4      2018-07-20 [1] CRAN (R 3.5.0)                   }
\CommentTok{#>  stringr       1.3.1      2018-05-10 [1] CRAN (R 3.5.0)                   }
\CommentTok{#>  testthat      2.0.1      2018-10-13 [1] CRAN (R 3.5.0)                   }
\CommentTok{#>  tibble        1.4.2      2018-01-22 [1] CRAN (R 3.5.0)                   }
\CommentTok{#>  tidyselect    0.2.5      2018-10-11 [1] CRAN (R 3.5.0)                   }
\CommentTok{#>  tinytex       0.9        2018-10-23 [1] CRAN (R 3.5.0)                   }
\CommentTok{#>  usethis       1.4.0      2018-08-14 [1] CRAN (R 3.5.0)                   }
\CommentTok{#>  withr         2.1.2      2018-03-15 [1] CRAN (R 3.5.0)                   }
\CommentTok{#>  xfun          0.4        2018-10-23 [1] CRAN (R 3.5.0)                   }
\CommentTok{#>  yaml          2.2.0      2018-07-25 [1] CRAN (R 3.5.0)                   }
\CommentTok{#> }
\CommentTok{#> [1] /Library/Frameworks/R.framework/Versions/3.5/Resources/library}
\end{Highlighting}
\end{Shaded}



\printbibliography

\newpage
\begin{appendices}
\appendixpage
\noappendicestocpagenum
\addappheadtotoc

\chapter{Supplementary materials for Chapter 2}

\renewcommand{\thetable}{\textbf{S\arabic{table}}}
\setcounter{table}{0}

% supplemental tables

\begin{table}
  \caption{Eight founder lines and their one-letter abbreviations.}
  \label{table-letters}
\begin{center}
\small
  \begin{tabular}{ c | c }
    \hline
    Founder allele & One-letter abbreviation \\ \hline
    A/J & A \\
    C57BL/6J & B \\
    129S1/SvImJ & C \\
    NOD/ShiLtJ & D\\
    NZO/H1LTJ & E\\
    Cast/EiJ & F\\
    PWK/PhJ & G\\
    WSB/EiJ & H\\
    \hline
  \end{tabular}

\end{center}
  \end{table}

\clearpage

  % latex table generated in R 3.5.1 by xtable 1.8-3 package
% Tue Nov 20 10:28:47 2018
\begin{table}
\caption{Both ``hot plate latency'' and ``percent time in light''
  demonstrate multiple QTL peaks with LOD scores above 5.}
  \label{table-peaks}
\begin{center}
\begin{tabular}{l|lrr}
  \hline
phenotype & chr & pos & LOD score \\
   \hline
percent time in light & 8 & 55.28 & 5.27 \\
 hot plate latency & 8 & 57.77 & 6.22 \\
 percent time in light & 9 & 36.70 & 5.42 \\
 hot plate latency & 9 & 46.85 & 5.22 \\
 percent time in light & 11 & 63.39 & 6.46 \\
 hot plate latency & 12 & 43.52 & 5.13 \\
 percent time in light & 15 & 15.24 & 5.67 \\
 hot plate latency & 19 & 47.80 & 5.48 \\
   \hline
\end{tabular}
\end{center}
\end{table}







\clearpage


% supplemental figures
\renewcommand{\thefigure}{\textbf{S\arabic{figure}}}
\setcounter{figure}{0}

\begin{figure}
\includegraphics[width = \textwidth]{../qtl2pleio-manuscript/Rmd/scatter.eps}
\caption{Scatter plot of ``hot plate latency'' against ``percent time in
  light'', after applying logarithm transformations and winsorizing
  both traits.}
\label{fig:scatter}
\end{figure}


\begin{figure}
\includegraphics{../qtl2pleio-manuscript/Rmd/genomewide_lods_10-22.eps}
\caption{Genome-wide QTL scan for percent time in light reveals
  multiple QTL, including one on Chromosome 8.}
\label{fig:genomewide10-22}
\end{figure}



\end{appendices}






\end{document}