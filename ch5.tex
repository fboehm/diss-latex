\chapter{Conclusions}

I've successfully developed a pleiotropy test for multiparental populations.
I discussed our new methods in Chapter~\ref{sec:testing}. In developing a pleiotropy test 
for multiparental populations, our novel contributions included accommodation of multiple 
alleles and incorporation of polygenic random effects to account for complicated patterns 
of relatedness. 
In Chapter~\ref{sec:applications}, we illustrated the test's use in three vignettes.
The first vignette compared pleiotropy testing with mediation analysis
in the dissection of expression trait QTL hotspots.
I learned that the pleiotropy test provides information about the
number of underlying QTL even when mediation analyses don't identify intermediates. 
Pleiotropy testing also serves as a useful screen before applying mediation analyses for a 
collection of putative intermediates.
The second vignette examined my test's power to detect separate QTL
in pairs of local expression traits.
I learned that both interlocus distance and univariate LOD scores
impact test statistic values. I was unable to find a strong relationship between allele effects patterns and statistical power.
In the last vignette, I applied my test to two gut microbiome-related traits.
From this analysis, I learned that the two traits share a pleiotropic QTL and, thus,
it is reasonable to conduct further causal modeling studies for these two traits.
Chapter~\ref{sec:software} demonstrates features of the \texttt{qtl2pleio} R package. 
\texttt{qtl2pleio} provides functions for multi-dimensional, multi-QTL scans. It also creates 
profile LOD plots and performs bootstrap tests to get p-values for the pleiotropy test 
statistics. 
This package uses the data structures in the R package \texttt{qtl2} \citep{broman2019rqtl2}.
I now conclude the thesis with brief discussions of limitations and future research.

\section{Limitations}

\subsection{A $d$-variate pleiotropy test}

Pleiotropy tests for two traits at a time have a valuable role in complex trait genetics. However, to 
fully use the tens of thousands of experimentally measured traits, we need to consider testing more than 
two traits at a time. Suppose that five traits map to a single region spanned by 100 markers. One might 
perform a series of $\binom{5}{2} = 10$ bivariate QTL scans and 10 pairwise tests for pleiotropy. Each 
bivariate scan would require $100^2 = 10,000$ model fits by generalized least squares. Alternatively, one 
could perform a d-variate QTL scan, with $d=5$ in this case. With the results of the d-variate scan, a 
variety of statistical hypotheses could be tested. For example, one could formulate a test for the null 
hypothesis that all five traits share a pleiotropic QTL against the alternative that the first two traits 
share a single QTL and the last three traits share a distinct, pleiotropic QTL. The d-variate scan over 
the 100-marker region, would require $100^d = 100^5 = 10$ billion model fits via generalized least 
squares. With distributed computing resources, including the resources at the University of Wisconsin's 
Center for High-throughput Computing, this is not an unreasonable volume of computing. The use of C++, instead of R, for 
generalized least squares calculations decreases the computing time for each model fit. 

My \texttt{qtl2pleio} R package contains code that performs $d$-variate, $d$-QTL scans for a 
genomic region. In this thesis, I set $d = 2$ for all analyses, yet the code and theory 
accommodate $d > 2$. The major hurdle in performing $d$-variate, $d$-QTL scans, as the above 
calculations suggest, is the computing time. Yet, even without modifying the current 
\texttt{qtl2pleio} code base, I can use computing clusters to complete multivariate, multi-QTL scans in reasonable time periods when $d$ is 3, 4, or 5. 

Before applying the $d$-variate pleiotropy test to experimental data, I would characterize its
statistical properties, like I did for the bivariate test in Chapter~\ref{sec:testing}. I would examine power and type I error rate for a variety of settings and distinct values of $d$. 

\subsection{Pleiotropy test power and allele effects patterns}

Based on findings from \citet{macdonald2007joint} and \citet{king2012genetic} I anticipated finding a stronger relationship between allele effects patterns and pleiotropy test power (Section~\ref{sec:power-analyses} and Figure~\ref{fig:cor}). \citet{macdonald2007joint} and \citet{king2012genetic} argue that similar allele effects patterns for two traits in multiparental populations favor pleiotropy over separate QTL when the QTL is bi-allelic. I would like to investigate this question with simulated traits in which we know the true genetic architecture. In Chapter~\ref{sec:power-analyses}, we used experimental data where I don't know the true number of QTL alleles. 

To address this question, I would perform a simulation study. I would first study pleiotropic (simulated) traits to see if they show evidence of similar allele effects, as measured by correlation between fitted values, when the QTL is bi-allelic. Because the Diversity Outbred mice have eight alleles at every locus, I want to consider the 22 partititions of eight alleles. The partition number (22 for eight objects) is the number of ways to form nonempty subsets with (unlabeled) objects. Because I'm concerned with merely the number of objects in each subset, rather than the labels of the objects, I only need to examine the 22 partitions. For example, one of the 22 partitions of eight objects is to have two subsets, where one subset has one allele and the other has seven alleles. A second partition of the 22 partitions of eight objects is to have eight subsets, with each subset containing exactly one allele.


\section{Future research}


\subsection{Selection bias}

Selection bias is a known concern in QTL studies \citep{lande1990efficiency}. Sometimes termed the ``Beavis effect'', after a researcher who described it in QTL studies \citep{beavis1991quantitative,beavis1994power}, selection bias arises when characterizing the QTL effect on the trait of interest. Given that a QTL is discovered, the estimated effect, in terms of proportion of trait variance explained, tends to overestimate the true effect \citep{broman2009guide}. Additionally, the number of detected QTL is biased downward in a genome-wide study \citep{beavis1998qtl}. Originally described in two-parent crosses, \citet{king2017beavis} found evidence for the Beavis effect in multiparental \emph{Drosophila melanogaster} populations. QTL studies in Diversity Outbred mice likely exhibit similar phenomena. \citet{xu2003theoretical} attributed the Beavis effect to the observation that QTL are only reported when the evidence in favor of a QTL exceeds a quantitative threshold. When \citet{xu2003theoretical} considered the appropriate truncated distributions, the experimental findings agreed with the theoretical expected results. 

In my studies, I identify traits of interest as pairs in which each shows sufficiently strong evidence of univariate QTL in a single genomic region. Identifying the traits of interest, then, is subject to the Beavis effect. To my knowledge, the impact of the Beavis effect on pleiotropy testing has not been studied. The direction of the Beavis effect on my pleiotropy test remains unclear at the time of this writing. Recall that our pleiotropy test statistic is the difference in log likelihood values under the alternative and under the null hypothesis. If there truly are two distinct QTL for a pair of traits, then it may be that the Beavis effect inflates the pleiotropy test statistic because the maximum log likelihood value in the two-dimensional grid may be more inflated than the maximum log likelihood along the diagonal (\emph{i.e.}, under pleiotropy). This question could be studied with simulated phenotypes using the genotypic data from \citet{keller2018genetic}.






\subsection{Determining significance thresholds for LOD difference and LOD difference proportion statistics}

One methodological question that arose in Chapter~\ref{sec:applications} is that of determining significance thresholds in mediation analyses. \citet{chick2016defining}, in a landmark investigation of mediation methods for sytems genetics studies, approximated the null LOD difference statistic distribution with an empirical distribution of sham intermediates. \citet{keller2018genetic} used an arbitrary threshold of 1.5 for declaring significant LOD difference statistics. To accommodate signals of differing strengths, I calculated LOD difference proportion statistics for the traits that Keller studied. However, I made no effort to determine a significance threshold for LOD difference proportion statistics. 

\subsection{Collapsing eight alleles to two to enhance power}

\citet{qtl2pattern} developed the R package \texttt{qtl2pattern} in which he collapses eight founder allele dosages into two ``pattern probabilities''. The assumption here is that, for some genomic regions, there may be only two alleles at each marker. If I can recognize the binary ``SNP distribution pattern'' for the genomic region, I could then collapse the eight alleles into two groups. For example, it may be that A, B, C, D, E, and F lines all have the same alleles at a sequence of markers, while G and H share a different set of alleles over the same interval. I would then partition the eight founder alleles into two groups, 1. ABCDEF and 2. GH, and determine the pattern probabilities. It might be that my pleiotropy test power would increase if I were to use the binary allele pattern probabilities instead of using the eight founder allele dosages. 

At the present time, this is an open research question. I would begin with a simulation study using Diversity Outbred mouse genotypic data from \citep{keller2018genetic}. With a collection of simulated traits, I would know the true allele patterns. I would then compare statistical power when I perform my pleiotropy test with each of the two encodings of genotype data, the eight founder allele dosages and the pattern probabilities.








\subsection{Multiple testing in mediation analysis}

I learned in Chapter~\ref{sec:applications} that my pleiotropy test is a useful screen before performing mediation analyses. One application of this fact is in reducing the number of mediation analyses when examining an expression QTL hotspot. Doing fewer mediation analyses leads to a less strict significance threshold for the LOD difference and related test statistics. I envision a procedure in which one performs a collection of pleiotropy tests pairing local expression traits with nonlocal expression traits, like I did in Section~\ref{sec:hotspot-dissection}. With the results of these tests, one may identify local expression trait - nonlocal expression trait pairs that are consistent with a single pleiotropic locus. Subsequent mediation analysis, for only those trait pairs that arise from pleiotropic loci, would be done. In this manner, I would reduce the number, and reduce the impact of inflated family-wise error rate for a collection of mediation analyses.



