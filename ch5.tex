\chapter{Conclusions}

I've successfully developed a pleiotropy test for multiparental populations.
I discussed our new methods in Chapter 2.
In Chapter 3, we illustrated the test's use in three vignettes.
The first vignette compared pleiotropy testing with mediation analysis
in the dissection of expression trait QTL hotspots.
I learned that the pleiotropy test provides information about the
number of underlying QTL even when mediation analyses don't identify intermediates. 
The second vignette examined my test's power to detect separate QTL
in pairs of local expression traits.
I learned that both interlocus distance and univariate LOD scores
impact test statistic values. 
In the last vignette, I applied my test to two gut microbiome-related traits.
From this analysis, I learned that the two traits share a pleiotropic QTL and, thus,
it is reasonable to conduct further causal modeling studies for these two traits.
Chapter 4 demonstrates features of the `qtl2pleio` R package.
This package implements my pleiotropy test and uses the data structures in the R package `qtl2` \citep{broman2019rqtl2}.

\section{A multivariate pleiotropy test}

Pleiotropy tests for two traits at a time have a valuable role in complex trait genetics. However, to fully use the tens of thousands of experimentally measured traits, we need to consider testing more than two traits at a time. Suppose that five traits map to a single region spanned by 100 markers. One might perform a series of $\binom{5}{2} = 10$ bivariate QTL scans and 10 pairwise tests for pleiotropy. Each bivariate scan would require $100^2 = 10,000$ model fits by generalized least squares. Alternatively, one could perform a d-variate QTL scan, with $d=5$ in this case. With the results of the d-variate scan, a variety of statistical hypotheses could be tested. For example, one could formulate a test for the null hypothesis that all five traits share a pleiotropic QTL against the alternative that the first two traits share a single QTL and the last three traits share a distinct, pleiotropic QTL. The d-variate scan over the 100-marker region, would require $100^d = 100^5 = 10$ billion model fits via generalized least squares. With distributed computing resources, including the resources at the University of Wisconsin's Center for High-throughput Computing, this is not an unreasonable volume of computing. The use of C++ for generalized least squares calculations decreases the computing time for each model fit. 

\section{X chromosome}



\section{Sensitivity analysis for unmeasured confounding in mediation analysis}


